%%%%%%%%%%%%%%%%%%%%%%%%%%%%%%%%%%%%%%%%%
% "ModernCV" CV and Cover Letter
% LaTeX Template
% Version 1.1 (9/12/12)
%
% This template has been downloaded from:
% http://www.LaTeXTemplates.com
%
% Original author:
% Xavier Danaux (xdanaux@gmail.com)
%
% License:
% CC BY-NC-SA 3.0 (http://creativecommons.org/licenses/by-nc-sa/3.0/)
%
% Important note:
% This template requires the moderncv.cls and .sty files to be in the same
% directory as this .tex file. These files provide the resume style and themes
% used for structuring the document.
%
%%%%%%%%%%%%%%%%%%%%%%%%%%%%%%%%%%%%%%%%%

%----------------------------------------------------------------------------------------
%   PACKAGES AND OTHER DOCUMENT CONFIGURATIONS
%----------------------------------------------------------------------------------------

\documentclass[11pt,a4paper,sans]{moderncv} % Font sizes: 10, 11, or 12; paper sizes: a4paper, letterpaper, a5paper, legalpaper, executivepaper or landscape; font families: sans or roman

\moderncvstyle{banking} % CV theme - options include: 'casual' (default), 'classic', 'oldstyle' and 'banking'
\moderncvcolor{blue} % CV color - options include: 'blue' (default), 'orange', 'green', 'red', 'purple', 'grey' and 'black'
\usepackage[margin=1in]{geometry} % Reduce document margins
\usepackage{lastpage}
\usepackage{mathabx}
\usepackage{libertine}
\cfoot{{\color{gray} Page \thepage\ of \pageref{LastPage}}}

\def\prd{{Phys. Rev.} D}
\def\PRL{{Phys.Rev.} Lett}
\def\apjl{{Astrophys. J.} Lett}
\def\apj{{Astrophys. J.}}
\def\CQG{{Class. Quantum Grav.}}
\def\aaps{{A\&AS}}
\def\pasj{{PASJ}}
\def\mnras{{MNRAS}} 
\def\aapr{{A\&ARv}}
\def\aap{{A\&A}}
\def\na{{New Astronomy}}
\def\ptp{{Progress of Theoretical Physics}}
\def\apjs{{ApJS}}
\def\araa{{ARA\&A}}
\def\ssr{{Space Sci. Rev.}} 

\usepackage{etoolbox}% http://ctan.org/pkg/etoolbox
\makeatletter
\newcommand*{\emailA}[1]{\def\@emailA{#1}}
\newcommand*{\emailB}[1]{\def\@emailB{#1}}
\patchcmd{\maketitle}% <cmd>
  {\ifthenelse{\isundefined{\@email}}{}{\addtomaketitle{\emailsymbol\emaillink{\@email}}}}% <search>
  {\ifthenelse{\isundefined{\@emailA}}{}{\addtomaketitle{\emailsymbol\emaillink{\@emailA}}}%
   \ifthenelse{\isundefined{\@emailB}}{}{\addtomaketitle{\emailsymbol\emaillink{\@emailB}}}}% <replace>
  {}{}% <success><failure>
\makeatother
 
%----------------------------------------------------------------------------------------
%   NAME AND CONTACT INFORMATION SECTION
%----------------------------------------------------------------------------------------

\firstname{Stephen} % Your first name
\familyname{Taylor} % Your last name

% All information in this block is optional, comment out any lines you don't need
\title{{\huge Curriculum Vitae}}
\address{TAPIR Group, MC $350$-$17$, California Institute of Technology}{1200 E. California Blvd, Pasadena, CA 91125, USA}
\phone[mobile]{+1 (626) 689-5832}
\emailA{Stephen.R.Taylor@jpl.nasa.gov}
%\emailB{steve.taylor1987@gmail.com}
\homepage{stevertaylor.github.io}
\social[github]{stevertaylor}
\social[linkedin][linkedin.com/in/stephen-taylor-a8164787]{stephen-taylor}


\newlength\listtripleitemmaincolumnwidth

\makeatletter
\renewcommand*{\recomputecvlengths}{%
  \setlength{\quotewidth}{0.65\textwidth}%
  \setlength{\maincolumnwidth}{\textwidth-\separatorcolumnwidth-\hintscolumnwidth}%
  \setlength{\listitemmaincolumnwidth}{\maincolumnwidth-\listitemsymbolwidth}%
  \setlength{\listdoubleitemmaincolumnwidth}{\maincolumnwidth-\listitemsymbolwidth-\separatorcolumnwidth-\listitemsymbolwidth}%
  \setlength{\listdoubleitemmaincolumnwidth}{0.66\listdoubleitemmaincolumnwidth}%
  \setlength\listtripleitemmaincolumnwidth{.66\listdoubleitemmaincolumnwidth}%
  \renewcommand{\headwidth}{\textwidth}%
  \setlength{\parskip}{0\p@}%
}
\makeatother

\newcommand{\cvdoublecolumn}[2]{%
  \cvline{}{%
  \begin{minipage}[t]{\listdoubleitemmaincolumnwidth}#1\end{minipage}%
  \hfill%
  \begin{minipage}[t]{\listdoubleitemmaincolumnwidth}#2\end{minipage}%
 }%
}

\newcommand{\cvtriplecolumn}[3]{%
  \cvline{}{%
  \begin{minipage}[t]{\listtripleitemmaincolumnwidth}#1\end{minipage}%
  \hfill%
  \begin{minipage}[t]{\listtripleitemmaincolumnwidth}#2\end{minipage}%
  \hfill%
  \begin{minipage}[t]{\listtripleitemmaincolumnwidth}#3\end{minipage}%
 }%
}

\newcommand{\cvreference}[9]{%
  \textbf{#1}\newline% Name
  \ifthenelse{\equal{#2}{}}{}{\addresssymbol~#2\newline}%
  \ifthenelse{\equal{#3}{}}{}{#3\newline}%
  \ifthenelse{\equal{#4}{}}{}{#4\newline}%
  \ifthenelse{\equal{#5}{}}{}{#5\newline}%
   \ifthenelse{\equal{#6}{}}{}{#6\newline}%
  \ifthenelse{\equal{#7}{}}{}{#7\newline}%
  \ifthenelse{\equal{#8}{}}{}{\emailsymbol~{\footnotesize \texttt{\emaillink[]{#8}}}\newline}
    \ifthenelse{\equal{#9}{}}{}{\phonesymbol~#9}}

%----------------------------------------------------------------------------------------
\renewcommand\UrlFont{\color{blue}\rmfamily}

\lhead{{\color{gray} \fontsize{10}{22}\mdseries\upshape{Stephen Taylor}}}
\rhead{{\color{gray} \fontsize{10}{10}\mdseries\upshape{Stephen.R.Taylor@jpl.nasa.gov}}}

\begin{document}
\hypersetup{urlcolor=blue,linkcolor=blue}
\makecvtitle % Print the CV title

%----------------------------------------------------------------------------------------
%   EDUCATION SECTION
%----------------------------------------------------------------------------------------
\vspace{-12mm}
\section{Education}
\cventry{2010--2014}{PhD (Astronomy)}{\underline{Institute of Astronomy, University of Cambridge}}{Cambridge, UK}{}{\textbf{Advisor:} Dr.~Jonathan R.~Gair; \textbf{Thesis Title:} \textit{Exploring The Cosmos With Gravitational Waves} }%\\ \textbf{Description:} A new Bayesian hierarchical modeling scheme is introduced to use compact-binary gravitational-wave standard sirens to infer the mass-distribution of the binary population, the progenitor star-formation rate, and cosmological parameters. Advanced pulsar-timing array techniques are developed to map the nanohertz gravitational-wave sky through a parametrized overlap-reduction function, and large accelerations to the Bayesian searches for single supermassive black-hole binaries are proposed.}  
\cventry{2006--2010}{MPhys ($1^\mathrm{st}$ Class), [ranked $1^\mathrm{st}$ in Jesus College, $4^\mathrm{th}$ across University]}{\underline{University of Oxford}}{Oxford, UK}{}{\textbf{Advisor:} Prof.~Steven Rawlings; \textbf{Thesis Title:} \textit{The Cosmic Evolution Of Black-hole Accretion}}

%----------------------------------------------------------------------------------------
%   WORK EXPERIENCE SECTION
%----------------------------------------------------------------------------------------
\vspace{-3.5mm}
\section{Professional Experience}
\cventry{2016--Present}{Caltech Postdoctoral Scholar (TAPIR group)}{\textsc{California Institute of Technology}}{Pasadena, USA}{}{} \vspace{-6mm}
\cventry{2014--2016}{Visiting scholar (TAPIR group)}{}{}{}{} 
\cventry{2014--2016}{NASA Postdoctoral Fellow}{\textsc{NASA Jet Propulsion Laboratory}}{Pasadena, USA}{}{} 
\cventry{2010--2014}{PhD candidate}{\textsc{Institute of Astronomy, University of Cambridge}}{Cambridge, UK}{}{}

%----------------------------------------------------------------------------------------
%   RESEARCH SECTION
%----------------------------------------------------------------------------------------

%\section{Research Interests}
%\cvitem{Primary interests}{gravitational-wave astronomy~$\bullet$~theoretical astrophysics~$\bullet$~massive black-hole binaries~$\bullet$~stellar-mass compact objects~$\bullet$~pulsar timing~$\bullet$~statistical inference}
%\cvitem{Secondary interests}{galaxy formation and evolution~$\bullet$~cosmology~$\bullet$~pulsar physics and demographics~$\bullet$~ionized interstellar medium}
%\cvitem{Specific interests}{pulsar-timing data-analysis for nanohertz gravitational-wave searches~$\bullet$~pulsar-timing noise characterization and mitigation~$\bullet$~waveform modeling for supermassive black-hole binary searches~$\bullet$~modeling final-parsec dynamics of supermassive black-hole binaries~$\bullet$~stochastic signal analysis strategies~$\bullet$~compact-binary demographics and population inference~$\bullet$~Bayesian hierarchical modeling}

%----------------------------------------------------------------------------------------
%   GRANTS & FUNDING
%----------------------------------------------------------------------------------------
\vspace{-3.5mm}
\section{Grants \& Funding}
\cvitem{Jun 2016}{\textbf{``New Directions and New Opportunities for NANOGrav Astrophysics'':} Awarded $\$11$k by the NANOGrav Physics Frontier Center to host a collaboration ``hack week'' in April $2017$.}{}

%----------------------------------------------------------------------------------------
%   AWARDS SECTION
%----------------------------------------------------------------------------------------
\vspace{-3.5mm}
\section{Honours \& Awards}
\cvitem{2015}{International Pulsar Timing Array (IPTA) Steering Committee Prize --- ``Honourable Mention''} %\vspace{-0.5mm}
\cvitem{2015}{Gravitational Wave International Committee (GWIC) Thesis Prize --- ``Honourable Mention''} %\vspace{-0.5mm}
\cvitem{2014}{NASA Postdoctoral Fellowship (JPL)} %\vspace{-0.5mm}
\cvitem{2013--2014}{Royal Astronomical Society Travel Awards --- [total exceeds $\pounds 1$k]} %\vspace{-0.5mm}
\cvitem{2012--2014}{Christ's College (Cambridge) Travel Grants [various; total exceeds $\pounds 1$k]} %\vspace{-0.5mm}
\cvitem{2010}{Science and Technology Facilities Council (STFC) --- full PhD studentship award} %\vspace{-0.5mm}
\cvitem{2008}{Examiner's Prize, Oxford Physics Speaking Competition} %\vspace{-0.5mm}
\cvitem{2007}{Oxford Physics department prize for laboratory work} %\vspace{-0.5mm}
\cvitem{2007--2010}{Undergraduate Scholar of Jesus College, Oxford} %\vspace{-0.5mm}
\cvitem{2006--2010}{Regularly awarded Oxford undergraduate departmental and college examination prizes}

%----------------------------------------------------------------------------------------
%   TEACHING SECTION
%----------------------------------------------------------------------------------------
\vspace{-3.5mm}
\section{Teaching Experience}
\cvitem{Jun--Aug 2016}{Co-supervisor of Caltech summer undergraduate student (Maya Fuller)} %\vspace{-0.5mm}
\cvitem{May 2016}{Guest Lecturer for Caltech Ph$237$ class ``Gravitational Waves''} %\vspace{-0.5mm}
\cvitem{Mar 2016}{Co-organiser of student workshop at NANOGrav Spring meeting} %\vspace{-0.5mm}
\cvitem{Sep 2015}{Lecturer for NANOGrav detection-group workshop at Caltech} %\vspace{-0.5mm}
\cvitem{Jun 2015}{Lecturer at ``CSI PTA'' Aspen summer workshop} %\vspace{-0.5mm}
\cvitem{2011--2013}{Supervisor for Cambridge Part II undergraduate students in \textsc{Relativity}} %\vspace{-0.5mm}
\cvitem{2011}{Prepared computing coursework for Cambridge Part II undergraduate students} %\vspace{-0.5mm}

%----------------------------------------------------------------------------------------
%   OUTREACH SECTION
%----------------------------------------------------------------------------------------
\vspace{-3.5mm}
\section{Professional Service} %\vspace{-1mm}

\subsection{Reviewer for international journals} %\vspace{-0.5mm}
\cvitem{}{Monthly Notices of the Royal Astronomical Society (MNRAS), Physical Review D (PRD)} %\vspace{-2mm}

\subsection{Conference and seminar organisation} %\vspace{-1mm}
\cvitem{Oct 2016}{Chair of SOC for NANOGrav Fall meeting at University of Illinois Urbana-Champaign} %\vspace{-0.5mm}
\cvitem{Mar 2016}{SOC and LOC member for NANOGrav Spring meeting at Caltech} %\vspace{-0.5mm}
\cvitem{Mar 2016}{Co-organiser of NANOGrav student workshop at Caltech} %\vspace{-0.5mm}
\cvitem{2015--2016}{Caltech TAPIR and LIGO postdoctoral lunch seminar series} %\vspace{-0.5mm}
\cvitem{Mar 2014}{SOC and LOC member for British Gravity meeting (BritGrav) at Cambridge, UK} %\vspace{-2mm}

\subsection{Professional affiliations} %\vspace{-1mm}
\textbf{North American Nanohertz Observatory for Gravitational-waves (NANOGrav)} [Full member]~$\bullet$~\textbf{European Pulsar Timing Array (EPTA)} [Member]~$\bullet$~\textbf{International Pulsar Timing Array (IPTA)} [Member]~$\bullet$~\textbf{American Physical Society (DGRAV)} [Member]~$\bullet$~\textbf{American Astronomical Society} [Member]~$\bullet$~\textbf{Royal Astronomical Society} [Fellow]

\vspace{-3.5mm}
\section{Outreach \& Media Engagement} %\vspace{-2mm}

\subsection{Outreach} %\vspace{-0.5mm}
\cvitem{2016}{Featured gravitational-wave expert at NASA's ``Ticket to Explore JPL'' event} %\vspace{-0.5mm}
\cvitem{2013}{Interactive presentation at Cambridge's Institute of Astronomy Open Day} %\vspace{-0.5mm}
\cvitem{2012--2014}{Presentation to prospective students (Institute of Astronomy graduate interviews)} %\vspace{-0.5mm}
\cvitem{2012}{Outreach talk at Institute of Astronomy public-observing evening: ``The Space Race''} %\vspace{-0.5mm}
\cvitem{2011}{Interactive presentation at Cambridge's Institute of Astronomy Open Day} %\vspace{-2mm}

\subsection{Press releases} %\vspace{-1mm}
\cvitem{Feb 2016}{Lead-author: {\color{blue} \href{http://www.jpl.nasa.gov/news/news.php?feature=5505}{``\textit{Pulsar Web Could Detect Low-Frequency Gravitational Waves}''}}} %\vspace{-0.5mm}
\cvitem{Apr 2016}{Collaboration: {\color{blue} \href{https://public.nrao.edu/news/pressreleases/2016-nanograv-sbr}{``\textit{Gravitational Wave Search Provides Insights into Galaxy Evolution and Mergers}''}}} %\vspace{-2mm}

%\subsection{Media coverage} \vspace{-1mm}
%\begin{itemize}
%\item Interviewed and quoted by \textit{Science} magazine: {\color{blue} \href{http://science.sciencemag.org/content/351/6278/1124}{``In Search of Spacetime Megawaves''}} by Daniel Clery%, Science  11 Mar 2016: Vol. 351, Issue 6278, pp. 1124-1125
%\item Quoted, with research featured in {\color{blue} \href{http://gizmodo.com/we-could-find-even-more-gravitational-waves-soon-with-p-1761021828}{\textit{Gizmodo}}}, {\color{blue} \href{https://www.engadget.com/2016/02/25/pulsars-gravitational-waves-black-holes/}{\textit{Engadget}}}, {\color{blue} \href{http://phys.org/news/2016-02-pulsar-web-low-frequency-gravitational.html}{\textit{Phys.org}}},  {\color{blue} \href{http://www.astronomy.com/news/2016/02/pulsar-web-could-detect-gravitational-waves}{\textit{Astronomy magazine}}}, {\color{blue} \href{http://www.universetoday.com/127562/the-future-of-gravitational-wave-astronomy-enhanced-ligo-pulsar-webs-space-interferometers-and-everything/}{\textit{Universe Today}}}
%\item Collaboration research featured in {\color{blue} \href{https://www.sciencedaily.com/releases/2016/04/160405122609.htm}{\textit{Science Daily}}}, {\color{blue} \href{https://astronomynow.com/2016/04/06/gravitational-wave-search-provides-insights-into-galaxy-mergers/}{\textit{Astronomy Now} (online)}}
%\end{itemize}

\vspace{-2.5mm}
\section{Publications}

$\bullet$ \textbf{$22$ peer-reviewed publications (of which $8$ are first-author) with $317$ citations, h-index $10$.} 

$\bullet$ Up-to-date metrics available at \url{https://scholar.google.com/citations?user=iN2djBMAAAAJ&hl=en}.

$\bullet$ $5$ key publications are indicated below, with most recent first.
\vspace{1pt}

\cvitem{{\color{teal}1}}{\textit{\textbf{S.~R.~Taylor}}, M.~Vallisneri, J.~A.~Ellis, C.~M.~F.~Mingarelli, T.~J.~W.~Lazio, and R.~van Haasteren. ``\textit{Are We There Yet? Time to Detection of Nanohertz Gravitational  Waves Based on Pulsar-timing Array Limits}''. \href{http://iopscience.iop.org/article/10.3847/2041-8205/819/1/L6/meta}{{\color{blue} Astrophys. J. Lett, 819:L6}} (2016). \textbf{[9 citations]}} 
\cvitem{{\color{teal}2}}{L.~Lentati, \textit{\textbf{S.~R.~Taylor}}, [and 34 others]. ``\textit{European Pulsar Timing Array limits on an isotropic stochastic gravitational-wave background}''. \href{http://mnras.oxfordjournals.org/content/453/3/2576}{{\color{blue} MNRAS, 453:2576--2598}} (2015). \textbf{[59 citations]}} 
\cvitem{{\color{teal}3}}{\textit{\textbf{S.~R.~Taylor}}, C.~M.~F.~Mingarelli, J.~R.~Gair, [and 32 others]. ``\textit{Limits on Anisotropy in the Nanohertz Stochastic Gravitational Wave Background}''. \href{http://journals.aps.org/prl/abstract/10.1103/PhysRevLett.115.041101}{{\color{blue} Phys.Rev. Lett, 115(4):041101}} (2015). \textbf{[14 citations]}} 
\cvitem{{\color{teal}4}}{\textit{\textbf{S.~R.~Taylor}} and J.~R.~Gair. ``\textit{Searching for anisotropic gravitational-wave backgrounds using pulsar timing arrays}''. \href{http://journals.aps.org/prd/abstract/10.1103/PhysRevD.88.084001}{{\color{blue} Phys. Rev. D, 88(8):084001}} (2013). \textbf{[31 citations]}} 
\cvitem{{\color{teal}5}}{\textit{\textbf{S.~R.~Taylor}}, J.~R.~Gair, and I.~Mandel. ``\textit{Cosmology using advanced gravitational-wave detectors alone}''. \href{http://journals.aps.org/prd/abstract/10.1103/PhysRevD.85.023535}{{\color{blue} Phys. Rev. D, 85(2):023535}} (2012). \textbf{[34 citations]}}

\vspace{-3.5mm}
\section{Presentations}

$\bullet$ \textbf{$33$ oral presentations (of which $13$ were invited), with $4$ conference leadership roles.}

$\bullet$ Recent presentations are available to view at {\color{blue} \href{https://speakerdeck.com/stevertaylor}{https://speakerdeck.com/stevertaylor}}.

%----------------------------------------------------------------------------------------
%   COMPUTER SKILLS SECTION
%----------------------------------------------------------------------------------------
%\vspace{-3.5mm}
%\section{Computing Skills}
%\textbf{OS:} Linux/UNIX, Windows~$\bullet$~\textbf{Programming:} C/C++, \textsc{Python}, UNIX shell scripting, HTML, GPU programming (CUDA C, PyCUDA)~$\bullet$~\textbf{Typography:} \LaTeX, Bibtex, Microsoft Office, Pages, OpenOffice~$\bullet$~\textbf{Scientific:} Mathematica, Matlab, \textsc{Python}

%----------------------------------------------------------------------------------------
%  REFERENCES
%----------------------------------------------------------------------------------------
\vspace{-3.5mm}
\section{References} 
Available upon request.
%\vspace{10pt}
%\cvdoublecolumn{\cvreference{Dr~Jonathan~R.~Gair [\textit{PhD advisor}]}
%    {Reader, School of Mathematics}
%    {James Clerk Maxwell Building}
%    {Peter Guthrie Tait Road}
%    {University of Edinburgh}
%    {Edinburgh UK, EH9 3FD}
%    {}
%    {J.Gair@ed.ac.uk}
%    {+44 (0) 131 650 4899}%
%    }
%    {\cvreference{Prof. Xavier Siemens}
%    {Chair, NANOGrav}
%    {Director, NANOGrav Physics Frontier Center}
%    {Associate Professor}
%    {Department of Physics}
%    {University of Wisconsin--Milwaukee}
%    {Milwaukee,  WI 53201}
%    {siemens@gravity.phys.uwm.edu}
%    {+1 (414) 229 6439}%
%    }
% 
%\vspace{15pt}
%\cvdoublecolumn{\cvreference{Dr~Michele~Vallisneri}
%    {Director's Fellow}
%    {Jet Propulsion Laboratory}
%    {California Institute of Technology}
%    {4800 Oak Grove Drive}
%    {Pasadena, CA 91109}
%    {}
%    {michele.vallisneri@jpl.nasa.gov}
%    {+1 (818) 395 9073}
%    }
%    {\cvreference{Prof. Alberto Vecchio}
%    {Head of Group}
%    {School of Physics \& Astronomy}
%    {University of Birmingham}
%    {Edgbaston}
%    {Birmingham UK, B15 2TT}
%    {}
%    {av@star.sr.bham.ac.uk}
%    {+44 (0) 1214 146447}
%    }
%    
%\vspace{15pt}
%\cvdoublecolumn{\cvreference{Dr~T.~Joseph~W.~Lazio}
%    {Chief Scientist}
%    {Interplanetary Network Directorate}
%    {Jet Propulsion Laboratory}
%    {California Institute of Technology}
%    {4800 Oak Grove Drive}
%    {Pasadena, CA 91109}
%    {Joseph.Lazio@jpl.nasa.gov}
%    {+1 (818) 354 4198}
%    }
%    {\cvreference{}
%    {}
%    {}
%    {}
%    {}
%    {}
%    {}
%    {}
%    {}
%    {}
%    }

\end{document}