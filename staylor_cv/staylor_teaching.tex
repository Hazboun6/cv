%%%%%%%%%%%%%%%%%%%%%%%%%%%%%%%%%%%%%%%%%
% "ModernCV" CV and Cover Letter
% LaTeX Template
% Version 1.1 (9/12/12)
%
% This template has been downloaded from:
% http://www.LaTeXTemplates.com
%
% Original author:
% Xavier Danaux (xdanaux@gmail.com)
%
% License:
% CC BY-NC-SA 3.0 (http://creativecommons.org/licenses/by-nc-sa/3.0/)
%
% Important note:
% This template requires the moderncv.cls and .sty files to be in the same
% directory as this .tex file. These files provide the resume style and themes
% used for structuring the document.
%
%%%%%%%%%%%%%%%%%%%%%%%%%%%%%%%%%%%%%%%%%

%----------------------------------------------------------------------------------------
%   PACKAGES AND OTHER DOCUMENT CONFIGURATIONS
%----------------------------------------------------------------------------------------

\documentclass[11pt,letterpaper,sans]{moderncv} % Font sizes: 10, 11, or 12; paper sizes: a4paper, letterpaper, a5paper, legalpaper, executivepaper or landscape; font families: sans or roman

\moderncvstyle{banking} % CV theme - options include: 'casual' (default), 'classic', 'oldstyle' and 'banking'
\moderncvcolor{blue} % CV color - options include: 'blue' (default), 'orange', 'green', 'red', 'purple', 'grey' and 'black'
%\usepackage[scale=0.8]{geometry} % Reduce document margins
\usepackage[margin=1in]{geometry}
\usepackage{lastpage}
%\setlength{\hintscolumnwidth}{3cm} % Uncomment to change the width of the dates column
%\setlength{\makecvtitlenamewidth}{10cm} % For the 'classic' style, uncomment to adjust the width of the space allocated to your name
\cfoot{Page \thepage\ of \pageref{LastPage}}

\def\prd{{Phys. Rev.} D}
\def\PRL{{Phys.Rev.} Lett}
\def\apjl{{Astrophys. J.} Lett}
\def\apj{{Astrophys. J.}}
\def\CQG{{Class. Quantum Grav.}}
\def\aaps{{A\&AS}}
\def\pasj{{PASJ}}
\def\mnras{{MNRAS}} 
\def\aapr{{A\&ARv}}
\def\aap{{A\&A}}
\def\na{{New Astronomy}}
\def\ptp{{Progress of Theoretical Physics}}
\def\apjs{{ApJS}}
\def\araa{{ARA\&A}}
\def\ssr{{Space Sci. Rev.}} 

\usepackage{etoolbox}% http://ctan.org/pkg/etoolbox
\makeatletter
\newcommand*{\emailA}[1]{\def\@emailA{#1}}
\newcommand*{\emailB}[1]{\def\@emailB{#1}}
\patchcmd{\maketitle}% <cmd>
  {\ifthenelse{\isundefined{\@email}}{}{\addtomaketitle{\emailsymbol\emaillink{\@email}}}}% <search>
  {\ifthenelse{\isundefined{\@emailA}}{}{\addtomaketitle{\emailsymbol\emaillink{\@emailA}}}%
   \ifthenelse{\isundefined{\@emailB}}{}{\addtomaketitle{\emailsymbol\emaillink{\@emailB}}}}% <replace>
  {}{}% <success><failure>
\makeatother
 
%----------------------------------------------------------------------------------------
%   NAME AND CONTACT INFORMATION SECTION
%----------------------------------------------------------------------------------------

\firstname{Stephen} % Your first name
\familyname{Taylor} % Your last name

% All information in this block is optional, comment out any lines you don't need
\title{\huge{Teaching Statement}}
\address{Jet Propulsion Laboratory, 4800 Oak Grove Drive}{Pasadena, CA 91109}
\phone[mobile]{+1 (626) 689-5832}
\emailA{Stephen.R.Taylor@jpl.nasa.gov}
%\emailB{steve.taylor1987@gmail.com}
\homepage{stevertaylor.github.io}
\social[github]{stevertaylor}
\social[linkedin][linkedin.com/in/stephen-taylor-a8164787]{stephen-taylor}

%----------------------------------------------------------------------------------------

\begin{document}

\makecvtitle % Print the CV title

\vspace{-6mm}

We remember our best and worst teachers, such is the indelible impact that teaching can have on us. A good teacher can make a struggling student achieve goals beyond their imagining, and can set a talented student on the road to greatness. A poor teacher can drag even the brightest student down. Instructing and supervising students is thus an incredible privilege that comes with many responsibilities that I take extremely seriously. %These students will form part of an unbroken chain of knowledge transfer --- it is my responsibility to pass on the discoveries of prior generations, whilst also teaching the students to think for themselves so that they can do the same for the next generation. 
I am particularly interested in ensuring that students become well-acquainted with foundational physical concepts through problem solving rather than highly prescribed learning. The former cultivates the student's ability to reason independently of their teacher and to intellectually mature, while the latter has little use beyond passing examinations. 
\vspace{1mm}

%Even during my undergraduate studies, I was benefiting from this kind of focused and personal teaching that inspires the greatest opportunities for intellectual growth. The Oxford system has tuition classes between a professor and one or two students, where they meet regularly to discuss homework problems and to recap lecture material. I was on the other side of this process when I tutored Cambridge Part II undergraduate students in relativity problem-solving classes during my own graduate studies. In many ways, this style of teaching is similar to the advisor and student relationship in research, where one must quickly assess the student's abilities and plan for the best ways for them to achieve their goals. 
%\vspace{2mm}

My first formal teaching experience was in graduate school, when I tutored Cambridge Part II undergraduate students in relativity problem-solving classes. The small size of these classes (at most three students) permitted focused and tailored learning to have the best opportunity for intellectual growth. In many ways, this style of teaching is similar to the advisor and student relationship in research, where one must quickly assess the student's abilities and plan for the best ways for them to achieve their goals. I began the academic year by asking the students why they were studying relativity, and what they hoped to get out of the course. From this I was able to gauge whether they were genuinely fascinated by the subject or if they simply needed to take the class. I had a responsibility to both types of students, but always strived to allow the latter group to see the mathematical beauty and physical power of relativity. 
\vspace{1mm}

Some physics (and relativity in particular) can seem abstract to new students, so in my classes I explained new concepts with as many physical examples as I could find, including why the ``twin paradox'' is not a paradox at all, and how fast one would need to travel to explain to the cops why they saw a green traffic light instead of a red one. The small classes allowed a conversational tone, and I encouraged more able students to attempt to explain challenging concepts to others, since a peer can sometimes have greater success in these scenarios. The classes were structured so that the students would receive back their graded homework problems at the start, and we would pick the most challenging problems to discuss. I required the students to tackle the problems in turn at the blackboard --- they could use their homework to explain the solution, but it was my goal that the students should develop their oral reasoning skills so that they would not simply mechanically crunch through the math. Students can sometimes assume that there is only one path to a solution, but I also used these classes to demonstrate alternative approaches and the links with other disciplines. 
\vspace{1mm}

My two benchmarks for determining whether the student's learning objectives were being met was: $(1)$ their performance in tackling the homework problems, and  $(2)$ their ability to coherently explain the solution to me and the rest of the class. These benchmarks assessed whether the student was familiar with the foundational concepts, could apply the concepts to problem solving, and had developed enough physical intuition and confidence with the material that they could explain it to others. I am proud to say that I am still in contact with one of my Cambridge students, who is now a graduate student at Caltech carrying out exoplanet research, and who has commented that he still refers back to my relativity notes when needed. Apart from making research breakthroughs, there are few occasions in academia to rival seeing a student directly benefit from your teaching and build on that for their own success. These classes were not without their share of challenges though. I had a student who seemed to struggle with almost every problem, to the point where she did not seem to be getting to grips with the material at all. However, after scheduling some refresher classes with her and reviewing the basic concepts of relativity, she showed a marked improvement.
\vspace{1mm}

%In some quarters, teaching is seen as lesser than research, since it is the latter that advances inquiry into the unknown. However, 
Arguably the greatest scientist of the $20^\mathrm{th}$ century once said: ``If you can't explain it simply, you don't understand it well enough''. Einstein saw that teaching organizes the mind and improves the practice of research by compelling you to break a subject down into easily-digestible logical steps. I have always found this to be true. %, whether it was helping out my fellow Oxford undergraduate physics students with a difficult new topic, teaching my Cambridge undergraduate students the beauty of special and general relativity, or instructing Caltech graduate students on the latest advances in gravitational-wave data-analysis strategies. 
By being removed from the trenches of research, students can sometimes ask insightful questions that spawn new lines of thought, and at the very least have compelled me to justify steps that I have taken in approaching a research problem. Such was the case when I recently co-supervised a Caltech summer undergraduate student. After a brief period of gaining familiarity with background material and analysis codes, she quickly developed her own intuition and understanding of the project goals to the stage where we will continue this fruitful collaboration during her sophomore year. Getting students involved in research at any early stage lets them see the scientific method in action, and helps to integrate the often separated concepts of learning and discovery.
\vspace{2mm}

%I don't believe that teaching students by rote can achieve anything other than them passing exams, with whatever spark of curiosity they initially had being slowly snuffed out in the process. My teaching philosophy follows a much more Socratic conversational style, where I try to equip the students with relevant topical knowledge (both the baseline textbook material and recent work in the literature), then engage them to see how far their own logic can take them in approaching unfamiliar problems. 
%As I mentioned before, I want my students to develop independent reasoning skills, since these are exceptionally important both for those who want to pursue their own academic careers and for those who will influence our society in other ways. Fostering independent thought as part of any lecture series or classes will not only keep the student interested and motivated for short-term examination purposes, but also cultivate longer-lasting critical faculties. This style was particularly useful when I recently co-supervised a Caltech summer undergraduate student. After a brief period of gaining familiarity with background material and analysis codes, she quickly developed her own intuition and understanding of the project goals to the stage where we will continue this fruitful collaboration during her sophomore year.
%\vspace{2mm}

Part of the importance of fostering critical thinking is for students to be able to tackle unfamiliar problems with familiar tools. %The great scientist and teacher (and personal hero of mine), Richard Feynman once said that sometimes people ``didn't even know what they knew'', so entrenched was their thinking in rote techniques. 
As we are now firmly in the era of gravitational-wave astronomy, my goal is to plan a class that shows students the links between inference strategies in LIGO, pulsar timing, and eLISA. The detectors and relevant gravitational-wave frequencies may greatly differ, but the data-analysis techniques are familiar in all cases. Pulsar timing has used deterministic-signal search techniques from LIGO, yet the former has developed far more sophisticated stochastic-signal techniques, with my own research recently contributing back to LIGO sky-mapping techniques. It is vital that new students appreciate the broad landscape of gravitational-wave astronomy, since they will likely benefit from discoveries in all three bands during the prime of their research careers. Instead of focusing on islands of rote-learned facts treated as specific to each detectors, I want this class to emphasize the deep bedrock of knowledge which connects the different disciplines. 
\vspace{2mm}

I have less experience lecturing large and introductory courses, but am comfortable teaching mathematical methods for the physical sciences, statistics for scientists, relativity, and gravitational-wave astrophysics. In all cases, I believe it is essential to tailor these courses to meet the diverse learning needs of the audience. Some will be there because they have a genuine passion for the subject, and are thus easier to motivate than those who may be taking the class to fulfill requirements. One group may go on to become new leaders in the academic research field, while the other group may go on to represent the scientific literacy of broader society. Regardless of why they are there, my responsibility as a teacher is to ensure that the core concepts are appreciated, problem solving skills are honed, and independent reasoning skills are developed. Meeting these objectives makes me a better researcher, cultivates the next generation of researchers, and gives those students who do not proceed down an academic career path a set of broader transferable skills.


\end{document}