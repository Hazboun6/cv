%%%%%%%%%%%%%%%%%%%%%%%%%%%%%%%%%%%%%%%%%
% "ModernCV" CV and Cover Letter
% LaTeX Template
% Version 1.1 (9/12/12)
%
% This template has been downloaded from:
% http://www.LaTeXTemplates.com
%
% Original author:
% Xavier Danaux (xdanaux@gmail.com)
%
% License:
% CC BY-NC-SA 3.0 (http://creativecommons.org/licenses/by-nc-sa/3.0/)
%
% Important note:
% This template requires the moderncv.cls and .sty files to be in the same
% directory as this .tex file. These files provide the resume style and themes
% used for structuring the document.
%
%%%%%%%%%%%%%%%%%%%%%%%%%%%%%%%%%%%%%%%%%

%----------------------------------------------------------------------------------------
%   PACKAGES AND OTHER DOCUMENT CONFIGURATIONS
%----------------------------------------------------------------------------------------

\documentclass[11pt,a4paper,sans]{moderncv} % Font sizes: 10, 11, or 12; paper sizes: a4paper, letterpaper, a5paper, legalpaper, executivepaper or landscape; font families: sans or roman

\moderncvstyle{banking} % CV theme - options include: 'casual' (default), 'classic', 'oldstyle' and 'banking'
\moderncvcolor{blue} % CV color - options include: 'blue' (default), 'orange', 'green', 'red', 'purple', 'grey' and 'black'
\usepackage[margin=1in]{geometry} % Reduce document margins
\usepackage{lastpage}
\usepackage{mathabx}
\usepackage{libertine}
\cfoot{{\color{gray} Page \thepage\ of \pageref{LastPage}}}

\def\prd{{Phys. Rev.} D}
\def\PRL{{Phys.Rev.} Lett}
\def\apjl{{Astrophys. J.} Lett}
\def\apj{{Astrophys. J.}}
\def\CQG{{Class. Quantum Grav.}}
\def\aaps{{A\&AS}}
\def\pasj{{PASJ}}
\def\mnras{{MNRAS}} 
\def\aapr{{A\&ARv}}
\def\aap{{A\&A}}
\def\na{{New Astronomy}}
\def\ptp{{Progress of Theoretical Physics}}
\def\apjs{{ApJS}}
\def\araa{{ARA\&A}}
\def\ssr{{Space Sci. Rev.}} 

\usepackage{etoolbox}% http://ctan.org/pkg/etoolbox
\makeatletter
\newcommand*{\emailA}[1]{\def\@emailA{#1}}
\newcommand*{\emailB}[1]{\def\@emailB{#1}}
\patchcmd{\maketitle}% <cmd>
  {\ifthenelse{\isundefined{\@email}}{}{\addtomaketitle{\emailsymbol\emaillink{\@email}}}}% <search>
  {\ifthenelse{\isundefined{\@emailA}}{}{\addtomaketitle{\emailsymbol\emaillink{\@emailA}}}%
   \ifthenelse{\isundefined{\@emailB}}{}{\addtomaketitle{\emailsymbol\emaillink{\@emailB}}}}% <replace>
  {}{}% <success><failure>
\makeatother
 
%----------------------------------------------------------------------------------------
%   NAME AND CONTACT INFORMATION SECTION
%----------------------------------------------------------------------------------------

\firstname{Stephen} % Your first name
\familyname{Taylor} % Your last name

% All information in this block is optional, comment out any lines you don't need
\title{\huge{Curriculum Vitae}}
\address{TAPIR Group, MC $350$-$17$, California Institute of Technology}{1200 E. California Blvd, Pasadena, CA 91125, USA}
\phone[mobile]{+1 (626) 689-5832}
\emailA{Stephen.R.Taylor@jpl.nasa.gov}
%\emailB{steve.taylor1987@gmail.com}
\homepage{stevertaylor.github.io}
\social[github]{stevertaylor}
\social[linkedin][linkedin.com/in/stephen-taylor-a8164787]{stephen-taylor}


\newlength\listtripleitemmaincolumnwidth

\makeatletter
\renewcommand*{\recomputecvlengths}{%
  \setlength{\quotewidth}{0.65\textwidth}%
  \setlength{\maincolumnwidth}{\textwidth-\separatorcolumnwidth-\hintscolumnwidth}%
  \setlength{\listitemmaincolumnwidth}{\maincolumnwidth-\listitemsymbolwidth}%
  \setlength{\listdoubleitemmaincolumnwidth}{\maincolumnwidth-\listitemsymbolwidth-\separatorcolumnwidth-\listitemsymbolwidth}%
  \setlength{\listdoubleitemmaincolumnwidth}{0.66\listdoubleitemmaincolumnwidth}%
  \setlength\listtripleitemmaincolumnwidth{.66\listdoubleitemmaincolumnwidth}%
  \renewcommand{\headwidth}{\textwidth}%
  \setlength{\parskip}{0\p@}%
}
\makeatother

\newcommand{\cvdoublecolumn}[2]{%
  \cvline{}{%
  \begin{minipage}[t]{\listdoubleitemmaincolumnwidth}#1\end{minipage}%
  \hfill%
  \begin{minipage}[t]{\listdoubleitemmaincolumnwidth}#2\end{minipage}%
 }%
}

\newcommand{\cvtriplecolumn}[3]{%
  \cvline{}{%
  \begin{minipage}[t]{\listtripleitemmaincolumnwidth}#1\end{minipage}%
  \hfill%
  \begin{minipage}[t]{\listtripleitemmaincolumnwidth}#2\end{minipage}%
  \hfill%
  \begin{minipage}[t]{\listtripleitemmaincolumnwidth}#3\end{minipage}%
 }%
}

\newcommand{\cvreference}[9]{%
  \textbf{#1}\newline% Name
  \ifthenelse{\equal{#2}{}}{}{\addresssymbol~#2\newline}%
  \ifthenelse{\equal{#3}{}}{}{#3\newline}%
  \ifthenelse{\equal{#4}{}}{}{#4\newline}%
  \ifthenelse{\equal{#5}{}}{}{#5\newline}%
   \ifthenelse{\equal{#6}{}}{}{#6\newline}%
  \ifthenelse{\equal{#7}{}}{}{#7\newline}%
  \ifthenelse{\equal{#8}{}}{}{\emailsymbol~{\footnotesize \texttt{\emaillink[]{#8}}}\newline}
    \ifthenelse{\equal{#9}{}}{}{\phonesymbol~#9}}

%----------------------------------------------------------------------------------------
\renewcommand\UrlFont{\color{blue}\rmfamily}

\lhead{{\color{gray} \fontsize{10}{22}\mdseries\upshape{Stephen Taylor}}}
\rhead{{\color{gray} \fontsize{10}{10}\mdseries\upshape{Stephen.R.Taylor@jpl.nasa.gov}}}

\begin{document}
\hypersetup{urlcolor=blue,linkcolor=blue}
\makecvtitle % Print the CV title

%----------------------------------------------------------------------------------------
%   OBJECTIVES
%----------------------------------------------------------------------------------------

\section{Objectives}
\begin{itemize}
\item My goal is to probe \textbf{compact objects} (neutron stars and black holes), their \textbf{environments}, and their \textbf{demographic properties} across the \textbf{full gravitational-wave spectrum}. 
\item I use the most sensitive \textbf{pulsar-timing} datasets to probe the dynamics and astrophysical environments of \textbf{supermassive black-hole binaries} in the nanohertz gravitational-wave band.
\item My future goals include performing full model-independent recoveries of the \textbf{demographics of LIGO-detected systems} to investigate their progenitor properties and evolutionary paths. 
\end{itemize}

%----------------------------------------------------------------------------------------
%   RESEARCH SECTION
%----------------------------------------------------------------------------------------

\section{Research Interests}
\cvitem{Primary interests}{gravitational-wave astronomy~$\bullet$~theoretical astrophysics~$\bullet$~massive black-hole binaries~$\bullet$~stellar-mass compact objects~$\bullet$~pulsar timing~$\bullet$~statistical inference}
\cvitem{Secondary interests}{galaxy formation and evolution~$\bullet$~cosmology~$\bullet$~pulsar physics and demographics~$\bullet$~ionised interstellar medium}
\cvitem{Specific interests}{pulsar-timing data-analysis for nanohertz gravitational-wave searches~$\bullet$~pulsar-timing noise characterisation and mitigation~$\bullet$~waveform modelling for supermassive black-hole binary searches~$\bullet$~modelling final-parsec dynamics of supermassive black-hole binaries~$\bullet$~stochastic signal analysis strategies~$\bullet$~compact-binary demographics and population inference~$\bullet$~Bayesian hierarchical modelling}

%----------------------------------------------------------------------------------------
%   EDUCATION SECTION
%----------------------------------------------------------------------------------------

\section{Education}
\cventry{2010--2014}{PhD (Astronomy)}{\underline{Institute of Astronomy, University of Cambridge}}{Cambridge, UK}{}{\textbf{Advisor:} Dr.~Jonathan R.~Gair; \textbf{Thesis Title:} \textit{Exploring The Cosmos With Gravitational Waves} \\ \textbf{Description:} A new Bayesian hierarchical modelling scheme is introduced to use compact-binary gravitational-wave standard sirens to infer the mass-distribution of the binary population, the progenitor star-formation rate, and cosmological parameters. Advanced pulsar-timing array techniques are developed to map the nanohertz gravitational-wave sky through a parametrised overlap-reduction function, and large accelerations to the Bayesian searches for single supermassive black-hole binaries are proposed.}  
\cventry{2006--2010}{MPhys ($1^\mathrm{st}$ Class), [ranked $1^\mathrm{st}$ in Jesus College, $4^\mathrm{th}$ across University]}{\underline{University of Oxford}}{Oxford, UK}{}{\textbf{Advisor:} Prof.~Steven Rawlings; \textbf{Thesis Title:} \textit{The Cosmic Evolution Of Black-hole Accretion}}

%----------------------------------------------------------------------------------------
%   WORK EXPERIENCE SECTION
%----------------------------------------------------------------------------------------

\section{Professional Experience}
\cventry{2016--Present}{Caltech Postdoctoral Scholar (TAPIR group)}{\textsc{California Institute of Technology}}{Pasadena, USA}{}{} \vspace{-6mm}
\cventry{2014--2016}{Visiting scholar (TAPIR group)}{}{}{}{} 
\cventry{2014--2016}{NASA Postdoctoral Fellow}{\textsc{NASA Jet Propulsion Laboratory}}{Pasadena, USA}{}{} 
\cventry{2010--2014}{PhD candidate}{\textsc{Institute of Astronomy, University of Cambridge}}{Cambridge, UK}{}{}

%----------------------------------------------------------------------------------------
%   GRANTS & FUNDING
%----------------------------------------------------------------------------------------

\section{Grants \& Funding}
\cvitem{Jun 2016}{\textbf{``New Directions and New Opportunities for NANOGrav Astrophysics''}}{Awarded $\$11$k by the NANOGrav Physics Frontier Center for a proposal on behalf of the collaboration's Astrophysics Working Group. Funding will ensure undergraduate/graduate students and outside experts can attend a sprint week in March $2017$ to advance several key areas of interest, and to achieve rapid progress on collaboration projects.}

%----------------------------------------------------------------------------------------
%   AWARDS SECTION
%----------------------------------------------------------------------------------------

\section{Honours \& Awards}
\cvitem{2015}{International Pulsar Timing Array (IPTA) Steering Committee Prize --- ``Honourable Mention''}
\cvitem{2015}{Gravitational Wave International Committee (GWIC) Thesis Prize --- ``Honourable Mention''}
\cvitem{2014}{NASA Postdoctoral Fellowship (JPL)}
\cvitem{2014}{Royal Astronomical Society Travel Award --- $\pounds 750$}
\cvitem{2013}{Royal Astronomical Society Travel Award --- $\pounds 700$}
\cvitem{2012--2014}{Christ's College (Cambridge) Travel Grants [various; total exceeds $\pounds 1$k]}
\cvitem{2010}{Science and Technology Facilities Council (STFC) --- full PhD studentship award}
\cvitem{2008}{Examiner's Prize, Oxford Physics Speaking Competition}
\cvitem{2007}{Oxford Physics department prize for laboratory work}
\cvitem{2007--2010}{Undergraduate Scholar of Jesus College, Oxford}
\cvitem{2006--2010}{Regularly awarded Oxford undergraduate departmental and college examination prizes}

%----------------------------------------------------------------------------------------
%   TEACHING SECTION
%----------------------------------------------------------------------------------------

\section{Teaching Experience}
\cvitem{Jun--Aug 2016}{Co-supervisor of Caltech summer undergraduate student (Maya Fuller)}
\cvitem{May 2016}{Guest Lecturer for Caltech Ph$237$ class ``Gravitational Waves''}
\cvitem{Mar 2016}{Co-organiser of student workshop at NANOGrav Spring meeting}
\cvitem{Sep 2015}{Lecturer for NANOGrav detection-group workshop at Caltech}
\cvitem{Jun 2015}{Lecturer at ``CSI PTA'' Aspen summer workshop}
\cvitem{2011--2013}{Supervisor for Cambridge Part II undergraduate students in \textsc{Relativity}}
\cvitem{2011}{Prepared computing coursework for Cambridge Part II undergraduate students}

%----------------------------------------------------------------------------------------
%   OUTREACH SECTION
%----------------------------------------------------------------------------------------

\section{Professional Service}

\subsection{Reviewer for international journals} 
\cvitem{}{Monthly Notices of the Royal Astronomical Society (MNRAS), Physical Review D (PRD)}

\subsection{Conference organisation} 
\cvitem{Oct 2016}{Chair of SOC for NANOGrav Fall meeting at University of Illinois Urbana-Champaign}
\cvitem{Mar 2016}{SOC and LOC member for NANOGrav Spring meeting at Caltech}
\cvitem{Mar 2016}{Co-organiser of NANOGrav student workshop at Caltech}
\cvitem{Mar 2014}{SOC and LOC member for British Gravity meeting (BritGrav) at Cambridge, UK}

\subsection{Seminar organisation} 
\cvitem{2015--2016}{Caltech TAPIR and LIGO postdoctoral lunch seminar series}

\subsection{Professional affiliations}
\begin{itemize}
\item North American Nanohertz Observatory for Gravitational-waves (NANOGrav), \textit{Full member}
\item European Pulsar Timing Array (EPTA), \textit{Member}
\item International Pulsar Timing Array (IPTA), \textit{Member}
\item American Physical Society (DGRAV), \textit{Member}
\item American Astronomical Society, \textit{Member}
\item Royal Astronomical Society, \textit{Fellow}
\end{itemize}

\section{Outreach \& Media Engagement}

\subsection{Outreach}
\cvitem{2016}{Featured gravitational-wave expert at NASA's ``Ticket to Explore JPL'' event}
\cvitem{2013}{Interactive presentation at Cambridge's Institute of Astronomy Open Day}
\cvitem{2012--2014}{Presentation to prospective students (Institute of Astronomy graduate interviews)}
\cvitem{2012}{Outreach talk at Institute of Astronomy public-observing evening: ``The Space Race''}
\cvitem{2011}{Interactive presentation at Cambridge's Institute of Astronomy Open Day}

\subsection{Press releases}
\cvitem{Feb 2016}{First-author research: {\color{blue} \href{http://www.jpl.nasa.gov/news/news.php?feature=5505}{``\textit{Pulsar Web Could Detect Low-Frequency Gravitational Waves}''}} (JPL press release)}
\cvitem{Apr 2016}{Collaboration research: {\color{blue} \href{https://public.nrao.edu/news/pressreleases/2016-nanograv-sbr}{``\textit{Gravitational Wave Search Provides Insights into Galaxy Evolution and Mergers}''}} (NRAO press release)}

\subsection{Media coverage}
\begin{itemize}
\item Interviewed and quoted by \textit{Science} magazine: {\color{blue} \href{http://science.sciencemag.org/content/351/6278/1124}{``In Search of Spacetime Megawaves''}} by Daniel Clery, Science  11 Mar 2016: Vol. 351, Issue 6278, pp. 1124-1125
\item Quoted, with research featured in {\color{blue} \href{http://gizmodo.com/we-could-find-even-more-gravitational-waves-soon-with-p-1761021828}{\textit{Gizmodo}}}, {\color{blue} \href{https://www.engadget.com/2016/02/25/pulsars-gravitational-waves-black-holes/}{\textit{Engadget}}}, {\color{blue} \href{http://phys.org/news/2016-02-pulsar-web-low-frequency-gravitational.html}{\textit{Phys.org}}},  {\color{blue} \href{http://www.astronomy.com/news/2016/02/pulsar-web-could-detect-gravitational-waves}{\textit{Astronomy magazine}}}, {\color{blue} \href{http://www.universetoday.com/127562/the-future-of-gravitational-wave-astronomy-enhanced-ligo-pulsar-webs-space-interferometers-and-everything/}{\textit{Universe Today}}}
\item Collaboration research featured in {\color{blue} \href{https://www.sciencedaily.com/releases/2016/04/160405122609.htm}{\textit{Science Daily}}}, {\color{blue} \href{https://astronomynow.com/2016/04/06/gravitational-wave-search-provides-insights-into-galaxy-mergers/}{\textit{Astronomy Now} (online)}}
\end{itemize}


\section{Publications}

$\bullet$ \textbf{$22$ peer-reviewed publications (of which $8$ are first-author) with $317$ citations, h-index $10$.} 

$\bullet$ Metrics available at \url{https://scholar.google.com/citations?user=iN2djBMAAAAJ&hl=en}.

$\bullet$ $5$ key publications are indicated below with preceding asterisks.

\subsection{Submitted \& In Preparation} 
\cvitem{}{\textbf{S.~R.~Taylor}, L.~Lentati, S.~Babak, P.~Brem, J.~R.~Gair, A.~Sesana, A.~Vecchio. ``\textit{All correlations must die: Assessing the significance of a stochastic gravitational-wave background in pulsar-timing arrays}''. Submitted to Physical Review D. {\color{blue} \href{https://arxiv.org/abs/1606.09180}{arXiv:1606.09180}}.}
\cvitem{}{\textbf{S.~R.~Taylor}, R.~ van Haasteren. ``\textit{Optimized anisotropic modelling of the nanohertz gravitational-wave sky with pulsar-timing arrays}''. (In Prep.)}
\cvitem{}{\textbf{S.~R.~Taylor}, J.~Simon, L.~Sampson. ``\textit{Bayesian model emulation for astrophysical inference of supermassive black-hole binaries with pulsar-timing arrays}''. (In Prep.)}

\subsection{Publications In Peer-reviewed International Journals} 
\cvitem{May 2016}{G.~Desvignes, R.~N.~Caballero, L.~Lentati, [and 40 others, including \textit{\textbf{S.~R.~Taylor}}]. ``\textit{High-precision timing of 42 millisecond pulsars with the European Pulsar Timing Array}''. \href{http://mnras.oxfordjournals.org/content/458/3/3341}{{\color{blue} MNRAS, 458:3341--3380}}.}
\cvitem{May 2016}{L.~Lentati, R.~M.~Shannon, W.~A.~Coles, [and 80 others, including \textit{\textbf{S.~R.~Taylor}}]. ``\textit{From spin noise to systematics: stochastic processes in the first International Pulsar Timing Array data release}''. \href{http://mnras.oxfordjournals.org/content/458/2/2161}{{\color{blue} MNRAS, 458:2161--2187}}.}
\cvitem{May 2016 }{J.~P.~W.~Verbiest, L.~Lentati, G.~Hobbs, [and 89 others, including \textit{\textbf{S.~R.~Taylor}}]. ``\textit{The International Pulsar Timing Array: First data release}''. \href{http://mnras.oxfordjournals.org/content/458/2/1267}{{\color{blue} MNRAS, 458:1267--1288}}.}
\cvitem{Apr 2016 }{Z.~Arzoumanian, A.~Brazier, S.~Burke-Spolaor, [and 48 others, including \textit{\textbf{S.~R.~Taylor}}]. ``\textit{The NANOGrav Nine-year Data Set: Limits on the Isotropic Stochastic Gravitational Wave Background}''. \href{http://iopscience.iop.org/article/10.3847/0004-637X/821/1/13/meta}{{\color{blue} Astrophys. J., 821:13}}.}
\cvitem{Apr 2016 }{R.~N.~Caballero, K.~J.~Lee, L.~Lentati, [and 36 others, including \textit{\textbf{S.~R.~Taylor}}]. ``\textit{The noise properties of 42 millisecond pulsars from the European Pulsar Timing Array and their impact on gravitational-wave searches}''. \href{http://mnras.oxfordjournals.org/content/457/4/4421}{{\color{blue} MNRAS, 457:4421--4440}}.}
${\color{red} \mathbf{^\Asterisk}}$\cvitem{Mar 2016 }{\textit{\textbf{S.~R.~Taylor}}, M.~Vallisneri, J.~A.~Ellis, C.~M.~F.~Mingarelli, T.~J.~W.~Lazio, and R.~van Haasteren. ``\textit{Are We There Yet? Time to Detection of Nanohertz Gravitational  Waves Based on Pulsar-timing Array Limits}''. \href{http://iopscience.iop.org/article/10.3847/2041-8205/819/1/L6/meta}{{\color{blue} Astrophys. J. Lett, 819:L6}}.}
\cvitem{Jan 2016 }{\textit{\textbf{S.~R.~Taylor}}, E.~A.~Huerta, J.~R.~Gair, and S.~T.~McWilliams. ``\textit{Detecting Eccentric Supermassive Black Hole Binaries with Pulsar Timing Arrays: Resolvable Source Strategies}''. \href{http://iopscience.iop.org/article/10.3847/0004-637X/817/1/70/meta}{{\color{blue} Astrophys. J., 817:70}}.}
\cvitem{Jan 2016 }{S.~Babak, A.~Petiteau, A.~Sesana, P.~Brem, P.~A.~Rosado, \textit{\textbf{S.~R.~Taylor}}, [and 30 others]. ``\textit{European Pulsar Timing Array limits on continuous gravitational waves from individual supermassive black hole binaries}''. \href{http://mnras.oxfordjournals.org/content/455/2/1665}{{\color{blue} MNRAS, 455:1665--1679}}.}
\cvitem{Nov 2015 }{J.~R.~Gair, J.~D.~Romano, and \textit{\textbf{S.~R.~Taylor}}. ``\textit{Mapping gravitational-wave backgrounds of arbitrary polarisation using pulsar timing arrays}''. \href{http://journals.aps.org/prd/abstract/10.1103/PhysRevD.92.102003}{{\color{blue} Phys. Rev. D, 92(10):102003}}.}
${\color{red} \mathbf{^\Asterisk}}$\cvitem{Nov 2015 }{L.~Lentati, \textit{\textbf{S.~R.~Taylor}}, C.~M.~F.~Mingarelli, [and 33 others]. ``\textit{European Pulsar Timing Array limits on an isotropic stochastic gravitational-wave background}''. \href{http://mnras.oxfordjournals.org/content/453/3/2576}{{\color{blue} MNRAS, 453:2576--2598}}.}
\cvitem{Sep 2015 }{E.~A.~Huerta, S.~T.~McWilliams, J.~R.~Gair, and \textit{\textbf{S.~R.~Taylor}}. ``\textit{Detection of eccentric supermassive black hole binaries with pulsar timing arrays: Signal-to-noise ratio calculations}''. \href{http://journals.aps.org/prd/abstract/10.1103/PhysRevD.92.063010}{{\color{blue} Phys. Rev. D, 92(6):063010}}.}
\cvitem{Aug 2015 }{J.~D.~Romano, \textit{\textbf{S.~R.~Taylor}}, N.~J.~Cornish, J.~Gair, C.~M.~F.~Mingarelli, and R.~van Haasteren. ``\textit{Phase-coherent mapping of gravitational-wave backgrounds using ground-based laser interferometers}'', \href{http://journals.aps.org/prd/abstract/10.1103/PhysRevD.92.042003}{{\color{blue} Phys. Rev. D, 92(4):042003}}.}
${\color{red} \mathbf{^\Asterisk}}$\cvitem{Jul 2015}{\textit{\textbf{S.~R.~Taylor}}, C.~M.~F.~Mingarelli, J.~R.~Gair, [and 32 others]. ``\textit{Limits on Anisotropy in the Nanohertz Stochastic Gravitational Wave Background}''. \href{http://journals.aps.org/prl/abstract/10.1103/PhysRevLett.115.041101}{{\color{blue} Phys.Rev. Lett, 115(4):041101}}.}
\cvitem{Mar 2015}{C.~J.~Moore, \textit{\textbf{S.~R.~Taylor}}, and J.~R.~Gair. ``\textit{Estimating the sensitivity of pulsar timing arrays}'', \href{http://iopscience.iop.org/article/10.1088/0264-9381/32/5/055004/meta}{{\color{blue} Classical and Quantum Gravity, 32(5):055004}}.}
\cvitem{Nov 2014}{\textit{\textbf{S.~R.~Taylor}}, J.~Ellis, and J.~Gair. ``\textit{Accelerated Bayesian model-selection and parameter-estimation in continuous gravitational-wave searches with pulsar-timing arrays}''. \href{http://journals.aps.org/prd/abstract/10.1103/PhysRevD.90.104028}{{\color{blue} Phys. Rev. D, 90(10):104028}}.}
\cvitem{Oct 2014}{J.~Gair, J.~D.~Romano, \textit{\textbf{S.~R.~Taylor}}, and C.~M.~F.~Mingarelli. ``\textit{Mapping gravitational-wave backgrounds using methods from CMB analysis: Application to pulsar timing arrays}''. \href{http://journals.aps.org/prd/abstract/10.1103/PhysRevD.90.082001}{{\color{blue} Phys. Rev. D, 90(8):082001}}.}
\cvitem{Aug 2014}{R.~M.~Shannon, S.~Chamberlin, N.~J.~Cornish, J.~A.~Ellis, C.~M.~F.~Mingarelli, D.~Perrodin, P.~Rosado, A.~Sesana, \textit{\textbf{S.~R.~Taylor}}, [and 14 others]. ``\textit{Summary of Session C1: pulsar timing arrays}''. \href{http://link.springer.com/article/10.1007\%2Fs10714-014-1765-4}{{\color{blue} General Relativity and Gravitation, 46:1765}}.}
${\color{red} \mathbf{^\Asterisk}}$\cvitem{Oct 2013}{\textit{\textbf{S.~R.~Taylor}} and J.~R.~Gair. ``\textit{Searching for anisotropic gravitational-wave backgrounds using pulsar timing arrays}''. \href{http://journals.aps.org/prd/abstract/10.1103/PhysRevD.88.084001}{{\color{blue} Phys. Rev. D, 88(8):084001}}.}
\cvitem{May 2013}{L.~Lentati, P.~Alexander, M.~P.~Hobson, \textit{\textbf{S.~R.~Taylor}}, J.~Gair, S.~T.~Balan, and R.~van Haasteren. ``\textit{Hyper-efficient model-independent Bayesian method for the analysis  of  pulsar timing data}''. \href{http://journals.aps.org/prd/abstract/10.1103/PhysRevD.87.104021}{{\color{blue} Phys. Rev. D, 87(10):104021}}.}
\cvitem{Feb 2013 }{\textit{\textbf{S.~R.~Taylor}}, J.~R.~Gair, and L.~Lentati. ``\textit{Weighing the evidence for a gravitational-wave background in the first International Pulsar Timing Array data challenge}''. \href{http://journals.aps.org/prd/abstract/10.1103/PhysRevD.87.044035}{{\color{blue} Phys. Rev. D, 87(4):044035}}.}
\cvitem{Jul 2012}{\textit{\textbf{S.~R.~Taylor}} and J.~R.~Gair. ``\textit{Cosmology with the lights off: Standard sirens in the Einstein Telescope era}''. \href{http://journals.aps.org/prd/abstract/10.1103/PhysRevD.86.023502}{{\color{blue} Phys. Rev. D, 86(2):023502}}.}
${\color{red} \mathbf{^\Asterisk}}$\cvitem{Jan 2012}{\textit{\textbf{S.~R.~Taylor}}, J.~R.~Gair, and I.~Mandel. ``\textit{Cosmology using advanced gravitational-wave detectors alone}''. \href{http://journals.aps.org/prd/abstract/10.1103/PhysRevD.85.023535}{{\color{blue} Phys. Rev. D, 85(2):023535}}.}

\section{Presentations}

$\bullet$ \textbf{$29$ oral presentations (of which $10$ were invited), with $4$ conference leadership roles.}

$\bullet$ Recent presentations are available to view at {\color{blue} \href{https://speakerdeck.com/stevertaylor}{https://speakerdeck.com/stevertaylor}}.

\subsection{Invited Talks} 
\cvitem{Oct 2016}{\textit{New data-analysis approaches for gravitational-wave searches with pulsar-timing arrays}, Montana State University seminar, Bozeman MT, USA}
\cvitem{Jul 2016}{\textit{New horizons in gravitational-wave astronomy with pulsar-timing arrays}, Armagh Observatory seminar, Armagh, UK}
\cvitem{Jul 2016}{\textit{Probing the final-parsec problem with pulsar-timing arrays}, Anton Pannekoek Institutt seminar, University of Amsterdam, Amsterdam, Netherlands}
\cvitem{Jul 2016}{\textit{Probing the final-parsec problem with pulsar-timing arrays}, Radboud University astrophysics seminar, Radboud, Netherlands}
\cvitem{Jun 2016}{\textit{Gravitational-wave data-analysis techniques for pulsar-timing arrays}, IPTA conference, Stellenbosch, South Africa}
\cvitem{Mar 2016}{\textit{Sources of nanohertz gravitational-waves for pulsar-timing array searches}, NANOGrav student workshop, Caltech, Pasadena CA, USA}
\cvitem{Dec 2015}{\textit{Prospects for near future detection and astrophysical inference with PTAs}, Gravitational-wave group seminar, University of Birmingham, UK}
\cvitem{Dec 2015}{\textit{Prospects for near future detection and astrophysical inference with PTAs}, Statistics group seminar (School of Mathematics), University of Edinburgh, UK}
\cvitem{Dec 2015}{\textit{Prospects for near future detection and astrophysical inference with PTAs}, CaJAGWR seminar, California Institute of Technology}
\cvitem{May 2013}{\textit{Searching For Anisotropic Gravitational-wave Backgrounds Using Pulsar Timing Arrays}, Albert Einstein Institute (AEI) GW seminar, Hanover}
\cvitem{Dec 2012}{\textit{Weighing the evidence for a gravitational-wave background}, Gravitational-wave group seminar, University of Birmingham, UK}


\subsection{Contributed Presentations} 
\cvitem{May 2016}{\textit{Carrying the physics of supermassive black-hole binary evolution into pulsar-timing array searches}, EPTA meeting, Bielefeld, Germany}
\cvitem{Apr 2016}{\textit{Are we there yet? Time to detection of nanohertz gravitational waves}, American Physical Society meeting, Salt Lake City UT, USA}
\cvitem{Mar 2016}{\textit{Carrying the physics of supermassive black-hole binary evolution into pulsar-timing array searches}, NANOGrav meeting, Caltech, Pasadena CA, USA}
\cvitem{Oct 2015}{\textit{Are we there yet? Time to detection of nanohertz gravitational waves}, NANOGrav meeting, McGill University, Montreal, Canada}
\cvitem{Jun 2015}{\textit{Eccentric supermassive black-hole binary signals in pulsar-timing data}, European Pulsar Timing Array meeting, Bonn, Germany}
\cvitem{Apr 2015}{\textit{Eccentric supermassive black-hole binary signals in pulsar-timing data}, American Physical Society meeting, Baltimore MD, USA}
\cvitem{Feb 2015}{\textit{Eccentric supermassive black-hole binary signals in pulsar-timing data}, NANOGrav meeting, Arecibo, Puerto Rico}
\cvitem{Jan 2015}{\textit{Exploring the cosmos with gravitational waves}, American Astronomical Society meeting, Seattle WA, USA}
\cvitem{Nov 2014}{\textit{EPTA constraints on gravitational-wave anisotropy}, European Pulsar Timing Array meeting, Cambridge, UK}
\cvitem{Jun 2014}{\textit{EPTA and IPTA searches for gravitational-wave background anisotropy}, International Pulsar Timing Array meeting, Banff, Canada}
\cvitem{May 2014}{\textit{EPTA limits on gravitational-wave anisotropy}, European Pulsar Timing Array meeting, Astron, Netherlands}
\cvitem{Oct 2013}{\textit{The pulsar-term in PTA continuous-wave searches: a blessing and a curse}, European Pulsar Timing Array meeting, Pula, Sardinia}
\cvitem{Jul 2013}{\textit{Probing anisotropy of the GW background with pulsar timing arrays}, 20th International Conference on General Relativity and Gravitation and 10th Amaldi Conference on Gravitational Waves, Warsaw}
\cvitem{Jun 2013}{\textit{The first PTA search pipeline for anisotropy in the GW background}, International Pulsar Timing Array meeting, Krabi, Thailand}
\cvitem{Apr 2013}{\textit{Searching For Anisotropic Gravitational-wave Backgrounds Using Pulsar Timing Arrays}, European Pulsar Timing Array meeting, l'Observatoire de Paris, Paris}
\cvitem{Feb 2013}{\textit{Weighing the evidence for a gravitational-wave background}, Institute of Astronomy seminar, University of Cambridge}
\cvitem{Nov 2012}{\textit{Weighing the evidence for a gravitational-wave background}, European Pulsar Timing Array meeting, Albert Einstein Institute (AEI), Potsdam, Germany}
\cvitem{Jun 2012}{\textit{Milestones in Spacetime: Double Neutron-Star Binaries as Gravitational-Wave Standard Sirens}, Institute of Astronomy seminar, University of Cambridge, UK}
\cvitem{Feb 2012}{\textit{Hubble without the Hubble: Cosmology using advanced gravitational-wave detectors alone}, Gravitational-Wave Meeting, Institut de Ci�ncies de l'Espai, Barcelona, Spain}


\subsection{Posters} 
\cvitem{Aug 2015}{\textit{Galactic environment effects on gravitational wave signals in pulsar timing arrays}, Postdoc Research Day, NASA Jet Propulsion Laboratory}
\cvitem{Aug 2012}{\textit{Cosmology without EM counterparts: Standard sirens in the advanced era and beyond}, Rattle and Shine, KITP Santa Barbara}
\cvitem{Dec 2011}{\textit{Cosmology using advanced gravitational-wave detectors alone}, Graduate Student Conference 2011, Cavendish Laboratory, University of Cambridge}

%----------------------------------------------------------------------------------------
%   COMPUTER SKILLS SECTION
%----------------------------------------------------------------------------------------

\section{Computing Skills}
\begin{itemize}
\item \textbf{OS:} Linux/UNIX, Windows
\item \textbf{Programming:} C/C++, \textsc{Python}, UNIX shell scripting, HTML, GPU programming (CUDA C, PyCUDA)
\item \textbf{Typography:} \LaTeX, Bibtex, Microsoft Office, Pages, OpenOffice 
\item \textbf{Scientific:} Mathematica, Matlab, \textsc{Python}
\end{itemize}

%----------------------------------------------------------------------------------------
%  REFERENCES
%----------------------------------------------------------------------------------------

\section{References} 
\vspace{15pt}
\cvdoublecolumn{\cvreference{Dr~Jonathan~R.~Gair [\textit{PhD advisor}]}
    {Reader, School of Mathematics}
    {James Clerk Maxwell Building}
    {Peter Guthrie Tait Road}
    {University of Edinburgh}
    {Edinburgh UK, EH9 3FD}
    {}
    {J.Gair@ed.ac.uk}
    {+44 (0) 131 650 4899}%
    }
    {\cvreference{Prof. Xavier Siemens}
    {Chair, NANOGrav}
    {Director, NANOGrav Physics Frontier Center}
    {Associate Professor}
    {Department of Physics}
    {University of Wisconsin--Milwaukee}
    {Milwaukee,  WI 53201}
    {siemens@gravity.phys.uwm.edu}
    {+1 (414) 229 6439}%
    }
 
\vspace{15pt}
\cvdoublecolumn{\cvreference{Dr~Michele~Vallisneri}
    {Director's Fellow}
    {Jet Propulsion Laboratory}
    {California Institute of Technology}
    {4800 Oak Grove Drive}
    {Pasadena, CA 91109}
    {}
    {michele.vallisneri@jpl.nasa.gov}
    {+1 (818) 395 9073}
    }
    {\cvreference{Prof. Alberto Vecchio}
    {Head of Group}
    {School of Physics \& Astronomy}
    {University of Birmingham}
    {Edgbaston}
    {Birmingham UK, B15 2TT}
    {}
    {av@star.sr.bham.ac.uk}
    {+44 (0) 1214 146447}
    }
    
\vspace{15pt}
\cvdoublecolumn{\cvreference{Dr~T.~Joseph~W.~Lazio}
    {Chief Scientist}
    {Interplanetary Network Directorate}
    {Jet Propulsion Laboratory}
    {California Institute of Technology}
    {4800 Oak Grove Drive}
    {Pasadena, CA 91109}
    {Joseph.Lazio@jpl.nasa.gov}
    {+1 (818) 354 4198}
    }
    {\cvreference{}
    {}
    {}
    {}
    {}
    {}
    {}
    {}
    {}
    {}
    }

\end{document}