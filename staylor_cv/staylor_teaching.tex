%%%%%%%%%%%%%%%%%%%%%%%%%%%%%%%%%%%%%%%%%
% "ModernCV" CV and Cover Letter
% LaTeX Template
% Version 1.1 (9/12/12)
%
% This template has been downloaded from:
% http://www.LaTeXTemplates.com
%
% Original author:
% Xavier Danaux (xdanaux@gmail.com)
%
% License:
% CC BY-NC-SA 3.0 (http://creativecommons.org/licenses/by-nc-sa/3.0/)
%
% Important note:
% This template requires the moderncv.cls and .sty files to be in the same
% directory as this .tex file. These files provide the resume style and themes
% used for structuring the document.
%
%%%%%%%%%%%%%%%%%%%%%%%%%%%%%%%%%%%%%%%%%

%----------------------------------------------------------------------------------------
%   PACKAGES AND OTHER DOCUMENT CONFIGURATIONS
%----------------------------------------------------------------------------------------

\documentclass[11pt,letterpaper,sans]{moderncv} % Font sizes: 10, 11, or 12; paper sizes: a4paper, letterpaper, a5paper, legalpaper, executivepaper or landscape; font families: sans or roman

\moderncvstyle{banking} % CV theme - options include: 'casual' (default), 'classic', 'oldstyle' and 'banking'
\moderncvcolor{blue} % CV color - options include: 'blue' (default), 'orange', 'green', 'red', 'purple', 'grey' and 'black'
%\usepackage[scale=0.8]{geometry} % Reduce document margins
\usepackage[margin=1in]{geometry}
%\setlength{\hintscolumnwidth}{3cm} % Uncomment to change the width of the dates column
%\setlength{\makecvtitlenamewidth}{10cm} % For the 'classic' style, uncomment to adjust the width of the space allocated to your name

\def\prd{{Phys. Rev.} D}
\def\PRL{{Phys.Rev.} Lett}
\def\apjl{{Astrophys. J.} Lett}
\def\apj{{Astrophys. J.}}
\def\CQG{{Class. Quantum Grav.}}
\def\aaps{{A\&AS}}
\def\pasj{{PASJ}}
\def\mnras{{MNRAS}} 
\def\aapr{{A\&ARv}}
\def\aap{{A\&A}}
\def\na{{New Astronomy}}
\def\ptp{{Progress of Theoretical Physics}}
\def\apjs{{ApJS}}
\def\araa{{ARA\&A}}
\def\ssr{{Space Sci. Rev.}} 

\usepackage{etoolbox}% http://ctan.org/pkg/etoolbox
\makeatletter
\newcommand*{\emailA}[1]{\def\@emailA{#1}}
\newcommand*{\emailB}[1]{\def\@emailB{#1}}
\patchcmd{\maketitle}% <cmd>
  {\ifthenelse{\isundefined{\@email}}{}{\addtomaketitle{\emailsymbol\emaillink{\@email}}}}% <search>
  {\ifthenelse{\isundefined{\@emailA}}{}{\addtomaketitle{\emailsymbol\emaillink{\@emailA}}}%
   \ifthenelse{\isundefined{\@emailB}}{}{\addtomaketitle{\emailsymbol\emaillink{\@emailB}}}}% <replace>
  {}{}% <success><failure>
\makeatother
 
%----------------------------------------------------------------------------------------
%   NAME AND CONTACT INFORMATION SECTION
%----------------------------------------------------------------------------------------

\firstname{Stephen} % Your first name
\familyname{Taylor} % Your last name

% All information in this block is optional, comment out any lines you don't need
\title{\huge{Teaching Statement}}
\address{Jet Propulsion Laboratory, 4800 Oak Grove Drive}{Pasadena, CA 91109}
\phone[mobile]{+1 (626) 689-5832}
\emailA{Stephen.R.Taylor@jpl.nasa.gov}
%\emailB{steve.taylor1987@gmail.com}
\homepage{stevertaylor.github.io}
\social[github]{stevertaylor}
\social[linkedin][linkedin.com/in/stephen-taylor-a8164787]{stephen-taylor}

%----------------------------------------------------------------------------------------

\begin{document}

\makecvtitle % Print the CV title

We remember our best and worst teachers, such is the indelible impact that teaching can have on us. A good teacher can make a struggling student achieve goals beyond their imagining, and can set a talented student on the road to greatness. A poor teacher can drag even the brightest student down. Instructing and supervising students is thus an incredible privilege that comes with many responsibilities that I take extremely seriously. These students will form part of an unbroken chain of knowledge transfer --- it is my responsibility to pass on the discoveries of prior generations, whilst also teaching the students to think for themselves so that they can do the same for the next generation.
\vspace{2mm}

In some quarters, teaching is seen as lesser than research, since it is the latter that advances inquiry into the unknown. However, arguably the greatest scientist of the $20^\mathrm{th}$ century once said: ``If you can't explain it simply, you don't understand it well enough''. Einstein saw that teaching organizes the mind and improves the practice of research by compelling you to break a subject down into easily-digestible logical steps. I have always found this to be true, whether it was helping out my fellow Oxford undergraduate physics students with a difficult new topic, teaching my Cambridge undergraduate students the beauty of special and general relativity, or instructing Caltech graduate students on the latest advances in gravitational-wave data-analysis strategies.
\vspace{2mm}

Even during my undergraduate studies, I was benefiting from the kind of focused and personal teaching that inspires the greatest opportunities for intellectual growth. The Oxford system has tuition classes between a professor and one or two students, where they meet regularly to discuss homework problems and to recap lecture material. I was on the other side of these classes as I taught relativity to Cambridge Part II undergraduate students during my own graduate studies. In many ways, this style of teaching is similar to the advisor and student relationship in research, where one must quickly assess the student's abilities and plan for the best ways for them to achieve their goals. I am proud to say that I am still in contact with one of my Cambridge students, who is now a graduate student at Caltech carrying out exoplanet research, and who has commented that he still refers back to my relativity notes when needed. Apart from making research breakthroughs, there are few occasions in academia to rival seeing a student directly benefit from your teaching and build on that for their own success.
\vspace{2mm}

I don't believe that teaching students by rote can achieve anything other than them passing exams, with whatever spark of curiosity they initially had being slowly snuffed out in the process. My teaching philosophy follows a much more Socratic conversational style, where I try to equip the students with relevant topical knowledge (both the baseline textbook material and recent work in the literature), then engage them to see how far their own logic can take them in approaching unfamiliar problems. Teaching students to think for themselves is often under-appreciated, but is exceptionally important both for those who want to pursue their own academic careers and for those who will influence our society in other ways. Thus, I believe that fostering independent thought as part of any lecture series or classes will not only keep the student interested and motivated for short-term examination purposes, but also cultivate longer-lasting critical faculties. This style was particularly useful when I recently co-supervised a Caltech summer undergraduate student. After a brief period of gaining familiarity with background material and analysis codes, she quickly developed her own intuition and understanding of the project goals to the stage where we will continue this fruitful collaboration during her sophomore year.
\vspace{2mm}

Part of the importance of fostering critical thinking is for students to be able to tackle unfamiliar problems with familiar tools. The great scientist and teacher (and personal hero of mine), Richard Feynman once said that sometimes people ``didn't even know what they knew'', so entrenched was their thinking in rote techniques. As we are now firmly in the era of gravitational-wave astronomy, I want to plan a class that shows students the links between inference strategies in LIGO, pulsar-timing, and eLISA. The detectors may be separate, and the relevant gravitational-wave frequencies separated by orders of magnitude, but the techniques are familiar in all cases. Pulsar-timing has used deterministic-signal search techniques from LIGO, yet the former has developed far more sophisticated stochastic-signal techniques which have recently fed back into LIGO. It is vital that new students appreciate the broad landscape of gravitational-wave astronomy, since they will likely benefit from discoveries in all three bands during the prime of their research careers. Instead of focusing on islands of rote-learned facts treated as specific to each detectors, I want this class to emphasize the deep bedrock of knowledge which connects the different disciplines. 
\vspace{2mm}

Additionally, while I have less experience teaching large introductory classes, I believe it is essential to tailor these courses to meet the diverse learning needs of the audience. Some will be there because they have a genuine passion for the subject, and are thus easier to motivate than those who may be taking the class to fulfill requirements. I have a responsibility to both types of students, since while one group may go on to become new leaders in the academic research field, the other group may go on to represent the scientific literacy of broader society. 

wide general good --> teaching strategies that manifest this good --> examples from specific classes --> evidence that the strategies were effective --> conclusion


\end{document}