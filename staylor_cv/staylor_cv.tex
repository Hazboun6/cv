\documentclass[11pt,letterpaper,sans]{moderncv}

\usepackage{graphicx}
\usepackage{longtable}
\usepackage{multirow}
\usepackage{textcomp}
\usepackage{units}
\usepackage{lineno}
\usepackage{rotating}
\usepackage{amssymb}
\usepackage{amsmath}
\usepackage[utf8]{inputenc}
\usepackage{longtable}
\usepackage{comment}
\usepackage{xcolor}
\usepackage[gen]{eurosym}


\moderncvstyle{banking}                           
\moderncvcolor{blue}    

\renewcommand{\familydefault}{\sfdefault}
\usepackage[top=1in,bottom=1in,left=1in,right=1in,bindingoffset=0cm]{geometry}
\setlength{\hintscolumnwidth}{3cm}
\usepackage{enumitem}
\setlist{nolistsep}
\usepackage{lastpage}
\usepackage{mathabx}
\cfoot{{\color{gray} Page \thepage\ of \pageref{LastPage}}}


\def\prd{{Phys. Rev.} D}
\def\PRL{{Phys.Rev.} Lett}
\def\apjl{{Astrophys. J.} Lett}
\def\apj{{Astrophys. J.}}
\def\CQG{{Class. Quantum Grav.}}
\def\aaps{{A\&AS}}
\def\pasj{{PASJ}}
\def\mnras{{MNRAS}} 
\def\aapr{{A\&ARv}}
\def\aap{{A\&A}}
\def\na{{New Astronomy}}
\def\ptp{{Progress of Theoretical Physics}}
\def\apjs{{ApJS}}
\def\araa{{ARA\&A}}
\def\ssr{{Space Sci. Rev.}} 

\newcommand{\cvreference}[9]{%
  \textbf{#1}\newline% Name
  \ifthenelse{\equal{#2}{}}{}{\addresssymbol~#2\newline}%
  \ifthenelse{\equal{#3}{}}{}{#3\newline}%
  \ifthenelse{\equal{#4}{}}{}{#4\newline}%
  \ifthenelse{\equal{#5}{}}{}{#5\newline}%
   \ifthenelse{\equal{#6}{}}{}{#6\newline}%
  \ifthenelse{\equal{#7}{}}{}{#7\newline}%
  \ifthenelse{\equal{#8}{}}{}{\emailsymbol~{\footnotesize \texttt{\emaillink[]{#8}}}\newline}
    \ifthenelse{\equal{#9}{}}{}{\phonesymbol~#9}}
 
%----------------------------------------------------------------------------------------
%   NAME AND CONTACT INFORMATION SECTION
%----------------------------------------------------------------------------------------

\firstname{Stephen} % Your first name
\familyname{Taylor} % Your last name

% All information in this block is optional, comment out any lines you don't need
\title{Curriculum Vitae}
\address{TAPIR Group, MC $350$-$17$, California Institute of Technology}{1200 E. California Blvd, Pasadena, CA 91125}
\phone[mobile]{+1~(626)~689-5832}
\email{steve.taylor@tapir.caltech.edu}
\homepage{stevertaylor.github.io}
\social[github]{stevertaylor}
\social[linkedin][linkedin.com/in/stephen-taylor-a8164787]{stephen-taylor $\;\;\bullet\;\;$ \today}

\lhead{{\color{gray} \fontsize{10}{22}\mdseries\upshape{Stephen Taylor}}}
\rhead{{\color{gray} \fontsize{10}{10}\mdseries\upshape{steve.taylor@tapir.caltech.edu}}}

%\definecolor{color1}{HTML}{1f77b4}
%\hypersetup{urlcolor=color1,linkcolor=color1}

\begin{document}
\makecvtitle 

%\vspace{-10mm}
%----------------------------------------------------------------------------------------
%   WORK EXPERIENCE SECTION
%----------------------------------------------------------------------------------------

\section{Research Experience}

\cventry{2017--Present}{NANOGrav PFC Senior Postdoctoral Fellow}{\textsc{California Institute of Technology}}{Pasadena, USA}{}{} \vspace{-6mm}
\cventry{}{Caltech Senior Postdoctoral Scholar (TAPIR group)}{}{}{}{}
\begin{tabular}{rcl}
&\hspace{0.4cm} &{\color{color1} $\circ\;\;$}{\textit{Support}}: NANOGrav Senior Postdoctoral Fellowship funded by an NSF Physics Frontier Center. \\
&\hspace{0.4cm} &  \hspace{0.4cm}This fellowship has funding and research-independence commensurate with NASA prize fellowships, \\
&\hspace{0.4cm} &  \hspace{0.4cm}and is the only senior fellowship that the NANOGrav PFC has ever awarded. \\
\end{tabular}\\ 
\vspace{1mm}

\cventry{2014--2017}{NASA Postdoctoral Fellow}{\textsc{NASA Jet Propulsion Laboratory}}{Pasadena, USA}{}{} \vspace{-6mm}
\cventry{}{Caltech Postdoctoral Scholar at JPL}{}{}{}{}
\begin{tabular}{rcl}
&\hspace{0.4cm} &{\color{color1} $\circ\;\;$}{\textit{Support}}: NASA Postdoctoral Program. This program provides promising early-career scientists \\
&\hspace{0.4cm} &  \hspace{0.4cm}the opportunity to carry out an independent research program that aligns with NASA's missions. \\
&\hspace{0.4cm} &  \hspace{0.4cm}Funding and research-independence commensurate with Einstein \& Hubble prize fellowships. \\
\end{tabular}\\ 
\vspace{1mm}

\cventry{2010--2014}{PhD candidate}{\textsc{University of Cambridge}}{Cambridge, UK}{}{}
\begin{tabular}{rcl}
&\hspace{0.4cm} &{\color{color1} $\circ\;\;$}{\textit{Support}}: Fully funded PhD studentship at the Institute of Astronomy, awarded by the UK \\
&\hspace{0.4cm} &  \hspace{0.4cm}Science \& Technology Facilities Council (STFC). Resulted in $7$ peer-reviewed publications.\\
\end{tabular}\\ 

%----------------------------------------------------------------------------------------
%   EDUCATION SECTION
%----------------------------------------------------------------------------------------
%\vspace{-4mm}
\section{Education}
\cventry{2010--2014}{PhD in Astronomy\vspace{1mm}}{University of Cambridge}{Cambridge, UK}{}{\textbf{Advisor:} Dr.~Jonathan R.~Gair; \textbf{Thesis Title:} \textit{Exploring The Cosmos With Gravitational Waves} \\ \textbf{Description:} A new Bayesian hierarchical population inference scheme is introduced to use LIGO compact-binary gravitational-wave standard sirens to infer the mass-distribution of the binary population, the progenitor star-formation rate, and cosmological parameters. Advanced pulsar-timing array techniques are developed to map the nanohertz gravitational-wave sky through a parametrized overlap-reduction function, and large accelerations to the Bayesian searches for resolvable supermassive black-hole binaries are implemented. \\  \textbf{\textit{Resulted in $7$ publications, and listed for GWIC Thesis Prize ``Honorable Mention"}}}  \vspace{2mm}

\cventry{2006--2010}{MPhys ($1^\mathrm{st}$ Class), [ranked $1^\mathrm{st}$ in Jesus College, $4^\mathrm{th}$ across University]\vspace{1mm}}{University of Oxford}{Oxford, UK}{}{\textbf{Advisor:} Prof.~Steven Rawlings; \textbf{Thesis Title:} \textit{The Cosmic Evolution Of Black-hole Accretion}}

%----------------------------------------------------------------------------------------
%   RESEARCH SECTION
%----------------------------------------------------------------------------------------
%\vspace{-4mm}
\section{Research Interests}
\cvitem{Primary interests}{gravitational-wave astronomy~$\bullet$~theoretical astrophysics~$\bullet$~massive black-hole binaries~$\bullet$~stellar-mass compact objects~$\bullet$~pulsar timing~$\bullet$~statistical inference}
\cvitem{Secondary interests}{galaxy formation and evolution~$\bullet$~cosmology~$\bullet$~pulsar physics and demographics~$\bullet$~ionized interstellar medium}
\cvitem{Specific interests}{Bayesian hierarchical modeling~$\bullet$~pulsar-timing data-analysis for nanohertz gravitational-wave searches~$\bullet$~compact-binary demographics and population inference~$\bullet$~pulsar-timing noise characterization and mitigation~$\bullet$~waveform modeling for supermassive black-hole binary searches~$\bullet$~modeling final-parsec dynamics of supermassive black-hole binaries~$\bullet$~stochastic signal analysis strategies}

%----------------------------------------------------------------------------------------
%  OBSERVING PROPOSALS
%----------------------------------------------------------------------------------------
%\vspace{-4mm}
\section{Observing Proposals}

\cvitemwithcomment{}{\textbf{``High impact MSPs for the International Pulsar Timing Array"}}{Nov 2016}
\begin{tabular}{rcl}
&\hspace{0.4cm} &{\color{color1} $\circ\;\;$}Green Bank Telescope, proposal GBT$17$A-$353$ \\
&\hspace{0.4cm} &{\color{color1} $\circ\;\;$}Status: awarded $21.0$ hours
\end{tabular} \\
\cvitemwithcomment{}{\textbf{``High impact MSPs for the International Pulsar Timing Array"}}{Sep 2016}
\begin{tabular}{rcl}
&\hspace{0.4cm} &{\color{color1} $\circ\;\;$}Arecibo Radio Telescope, proposal P$3133$ \\
&\hspace{0.4cm} &{\color{color1} $\circ\;\;$}Status: awarded $32.5$ hours
\end{tabular} \\

%----------------------------------------------------------------------------------------
%   GRANTS & FUNDING
%----------------------------------------------------------------------------------------
%\cvitem{Jun 2016}{\textbf{``New Directions and New Opportunities for NANOGrav Astrophysics''}}{Awarded $\$11$k by the NANOGrav Physics Frontier Center for a proposal on behalf of the collaboration's Astrophysics Working Group. Funding ensured undergraduate/graduate students and outside experts could attend a sprint week in April $2017$ to advance several key areas of interest, and to achieve rapid progress on collaboration projects.}
%\vspace{-4mm}
\section{Grants, Funding \& Awards}

\cvitemwithcomment{}{\textbf{Marie Sk\l{}odowska-Curie Incoming Fellowship, University of Birmingham [declined]}}{2017}
\begin{tabular}{rcl}
&\hspace{0.4cm} &{\color{color1} $\circ\;\;$}Proposal ``GravPANTHER". Grant exceeds \euro$150$k.
\end{tabular} \\
\cvitemwithcomment{}{\textbf{OzGrav ARC Centre of Excellence Level B Fellowship, Swinburne \& Monash [declined]}}{2017}
\begin{tabular}{rcl}
&\hspace{0.4cm} &{\color{color1} $\circ\;\;$}$7$-year senior research fellowship at Australian lecturer-level salary 
\end{tabular} \\
\cvitemwithcomment{Proposal (Co-I)}{\textbf{``Exploring the frontiers of the nHz GW spectrum with PTAs''}}{2017}
\begin{tabular}{rcl}
&\hspace{0.4cm} &{\color{color1} $\circ\;\;$}Awarded $1$M CPU hours ($\$600$k grant) on the Blue Waters supercomputer at NCSA, \\
&\hspace{0.4cm} &  \hspace{0.4cm}University of Illinois at Urbana Champaign. \\
&\hspace{0.4cm} &{\color{color1} $\circ\;\;$}The project developed a remote interactive interface for pulsar-timing GW searches \\
&\hspace{0.4cm} &  \hspace{0.4cm}on the Blue Waters HPC architecture, using a custom-built Docker software container.
\end{tabular} \\
\cvitemwithcomment{Proposal (PI)}{\textbf{``New Directions and New Opportunities for NANOGrav Astrophysics''}}{2016}
\begin{tabular}{rcl}
&\hspace{0.4cm} &{\color{color1} $\circ\;\;$}Awarded $\$11$k by the NANOGrav Physics Frontier Center for a proposal on behalf \\
&\hspace{0.4cm} &  \hspace{0.4cm}of the collaboration's Astrophysics Working Group. \\
&\hspace{0.4cm} &{\color{color1} $\circ\;\;$}Funding ensured undergraduate/graduate students and outside experts could attend \\
&\hspace{0.4cm} &  \hspace{0.4cm}a sprint week at West Virginia University in April $2017$ to advance several key areas \\
&\hspace{0.4cm} &  \hspace{0.4cm}of interest, and to achieve rapid progress on collaboration projects. 
\end{tabular} \\
\cvitemwithcomment{}{\textbf{International Pulsar Timing Array Steering Committee Prize --- ``Honorable Mention''}}{2016} \vspace{-0.1cm}
\cvitemwithcomment{}{\textbf{Gravitational Wave International Committee Thesis Prize --- ``Honorable Mention''}}{2016} \vspace{-0.1cm}
\cvitemwithcomment{}{\textbf{NASA Postdoctoral Fellowship at Jet Propulsion Laboratory}}{2014} \vspace{-0.1cm}
\cvitemwithcomment{}{\textbf{Royal Astronomical Society Travel Awards [various; total $\pounds 1450$]}}{2013--2014} \vspace{-0.1cm}
\cvitemwithcomment{}{\textbf{Christ's College (Cambridge) Travel Grants [various; total exceeds $\pounds 1$k]}}{2012--2014} \vspace{-0.1cm}
\cvitemwithcomment{}{\textbf{Science and Technology Facilities Council (STFC) --- full PhD studentship award}}{2010} \vspace{-0.1cm}
\cvitemwithcomment{}{\textbf{Examiner's Prize, Oxford Physics Speaking Competition}}{2008} \vspace{-0.1cm}
\cvitemwithcomment{}{\textbf{Oxford Physics department prize for outstanding laboratory work}}{2007} \vspace{-0.1cm}
\cvitemwithcomment{}{\textbf{Undergraduate Scholarship in Jesus College, Oxford}}{2007--2010} \vspace{-0.1cm}
\cvitemwithcomment{}{\textbf{Oxford Physics Dept.\ and Jesus College examination prizes [total exceeds $\pounds 1$k]}}{2006--2010}

%----------------------------------------------------------------------------------------
%   TEACHING SECTION
%----------------------------------------------------------------------------------------
%\vspace{-4mm}
\section{Teaching \& Mentoring}

%\textbf{\textcolor{black}{Undergraduate student mentoring:}} 
\subsection{Undergraduate student mentoring}

\cvitemwithcomment{}{\hspace{0.4cm}${\color{color1} \circ}\;$ \textit{Maya Fuller}, Caltech}{2016--2017} \vspace{-0.1cm}
\hspace{0.71cm} Mentor for learning Bayesian statistical inference methods and scientific computing.

\cvitemwithcomment{}{\hspace{0.4cm}${\color{color1} \circ}\;$ \textit{Maya Fuller}, Caltech}{Summer 2016}\vspace{-0.1cm}
\hspace{0.71cm} Co-supervisor for summer research project: \textit{``Bayesian estimates of pulsar-timing \\ \vspace{-0.1cm}
\hspace{0.6cm} array sensitivity to nanohertz gravitational-waves''}. \vspace{0.1cm}

\cvitemwithcomment{}{\hspace{0.4cm}${\color{color1} \circ}\;$ \textit{Jacob Turner}, Oberlin College}{Summer 2016}\vspace{-0.1cm}
\hspace{0.71cm} Co-mentor for summer research project: \textit{``Correcting for interstellar scattering \\ \vspace{-0.1cm}
\hspace{0.6cm} delays in pulsar-timing''}. \vspace{0.1cm}

%\textbf{\textcolor{black}{Lecturing:}} 
\subsection{Lecturing}

\cvitemwithcomment{}{\hspace{0.4cm}${\color{color1} \circ}\;$ Instructor at NANOGrav GW Detection workshop at Caltech}{Mar 2017}\vspace{-0.1cm}

\cvitemwithcomment{}{\hspace{0.4cm}${\color{color1} \circ}\;$ Lecturer at NANOGrav Fall 2016 student workshop at UIUC}{Oct 2016}\vspace{-0.1cm}

\cvitemwithcomment{}{\hspace{0.4cm}${\color{color1} \circ}\;$ Guest Lecturer for Caltech Ph$237$ class ``Gravitational Waves''}{May 2016}\vspace{-0.1cm}

\cvitemwithcomment{}{\hspace{0.4cm}${\color{color1} \circ}\;$ Lecturer and co-organizer at NANOGrav Spring 2016 student workshop at Caltech}{Mar 2016}\vspace{-0.1cm}

\cvitemwithcomment{}{\hspace{0.4cm}${\color{color1} \circ}\;$ Instructor at NANOGrav GW Detection workshop at Caltech}{Sep 2016}\vspace{-0.1cm}

\cvitemwithcomment{}{\hspace{0.4cm}${\color{color1} \circ}\;$ Lecturer at ``CSI PTA'' Aspen Center for Physics workshop}{Jun 2015}

%\textbf{\textcolor{black}{Teaching assistant:}} 
\subsection{Teaching assistant}

\cvitemwithcomment{}{\hspace{0.4cm}${\color{color1} \circ}\;$ University of Cambridge, Part II Relativity ($3$rd-year undergraduate class)}{2011--2013}\vspace{-0.1cm}
\hspace{0.71cm} \textit{Set and graded homework problems, mentored students, led recitation classes.}

\cvitemwithcomment{}{\hspace{0.4cm}${\color{color1} \circ}\;$ University of Cambridge, Part II Astronomy (computing coursework)}{2011}\vspace{-0.1cm}
\hspace{0.71cm} \textit{Prepared materials for $3$rd-year undergraduate class.} \vspace{-0.1cm}

%----------------------------------------------------------------------------------------
%   OUTREACH SECTION
%----------------------------------------------------------------------------------------
%\vspace{-4mm}
\section{Professional Service}

\subsection{Reviewer for international journals} 
\cvitem{}{The Astrophysical Journal, Monthly Notices of the Royal Astronomical Society, Physical Review D, Physical Review Letters}

\subsection{Conference organization} 
\cvitemwithcomment{}{\hspace{0.4cm}${\color{color1} \circ}\;$ SOC member for International Pulsar Timing Array meeting in S\`evres, France}{Jul 2017} \vspace{-0.1cm}
\cvitemwithcomment{}{\hspace{0.4cm}${\color{color1} \circ}\;$ Organizer of NANOGrav Astrophysics sprint week at West Virginia Univesity}{Apr 2017} \vspace{-0.1cm}
\cvitemwithcomment{}{\hspace{0.4cm}${\color{color1} \circ}\;$ Co-organizer of NANOGrav Detection workshop at Caltech}{Mar 2017} \vspace{-0.1cm}
\cvitemwithcomment{}{\hspace{0.4cm}${\color{color1} \circ}\;$ Co-organizer of NANOGrav Hackathon at NCSA}{Oct 2017} \vspace{-0.1cm}
\cvitemwithcomment{}{\hspace{0.4cm}${\color{color1} \circ}\;$ Chair of SOC for NANOGrav Fall meeting at University of Illinois Urbana Champaign}{Oct 2016} \vspace{-0.1cm}
\cvitemwithcomment{}{\hspace{0.4cm}${\color{color1} \circ}\;$ SOC and LOC member for NANOGrav Spring meeting at Caltech}{Mar 2016} \vspace{-0.1cm}
\cvitemwithcomment{}{\hspace{0.4cm}${\color{color1} \circ}\;$ Co-organizer of NANOGrav student workshop at Caltech}{Mar 2016} \vspace{-0.1cm}
\cvitemwithcomment{}{\hspace{0.4cm}${\color{color1} \circ}\;$ SOC and LOC member for British Gravity meeting (BritGrav) at Cambridge, UK}{Mar 2014} \vspace{-0.1cm}

\subsection{Seminar organization} 
\cvitemwithcomment{}{\hspace{0.4cm}${\color{color1} \circ}\;$ Caltech TAPIR and LIGO postdoctoral lunch seminar series}{2015--2016} \vspace{-0.1cm}

\subsection{Code \& data sharing} 
\cvitemwithcomment{}{\hspace{0.4cm}${\color{color1} \circ}\;$ Developer of python open-source GW analysis software \textsc{NX01}}{2015--Present} \vspace{-0.1cm}
\hspace{0.71cm} {\color{color1} \href{https://github.com/stevertaylor/NX01}{https://github.com/stevertaylor/NX01}}

\cvitemwithcomment{}{\hspace{0.4cm}${\color{color1} \circ}\;$ Release of code and library of simulated stochastic gravitational-wave spectra}{2017} \vspace{-0.1cm}
\hspace{0.71cm} Associated with Taylor et al., PRL, 118, 181102 (2017) \\ \vspace{-0.1cm}
\hspace{0.6cm} {\color{color1} \href{https://github.com/stevertaylor/gw_pta_emulator}{https://github.com/stevertaylor/gw$\_$pta$\_$emulator}}

\subsection{Professional affiliations}
\cvitemwithcomment{}{\hspace{0.4cm}${\color{color1} \circ}\;$ North American Nanohertz Observatory for Gravitational-waves (NANOGrav), \textit{Full member}}{} \vspace{-0.1cm}
\cvitemwithcomment{}{\hspace{0.4cm}${\color{color1} \circ}\;$ European Pulsar Timing Array (EPTA), \textit{Member}}{} \vspace{-0.1cm}
\cvitemwithcomment{}{\hspace{0.4cm}${\color{color1} \circ}\;$ International Pulsar Timing Array (IPTA), \textit{Member}}{} \vspace{-0.1cm}
\cvitemwithcomment{}{\hspace{0.4cm}${\color{color1} \circ}\;$ American Physical Society (DGRAV), \textit{Member}}{} \vspace{-0.1cm}
\cvitemwithcomment{}{\hspace{0.4cm}${\color{color1} \circ}\;$ Royal Astronomical Society, \textit{Fellow}}{} \vspace{-0.1cm}

%\vspace{-4mm}
\section{Outreach \& Media Engagement}

\subsection{Outreach}
\cvitemwithcomment{}{\hspace{0.4cm}${\color{color1} \circ}\;$ Astronomy expert at Caltech's {\color{color1} \href{http://www.caltech.edu/content/solar-eclipse-viewing-party}{``Solar Eclipse Viewing Party''}}}{2017} \vspace{-0.1cm}
\cvitemwithcomment{}{\hspace{0.4cm}${\color{color1} \circ}\;$ Featured speaker at West Virginia University's {\color{color1} \href{http://einstein.wvu.edu/featurettes/einstein-lectures}{``Celebrating Einstein''}} public-lecture series}{2017} \vspace{-0.1cm}
\cvitemwithcomment{}{\hspace{0.4cm}${\color{color1} \circ}\;$ Featured gravitational-wave expert at NASA's {\color{color1} \href{https://www.jpl.nasa.gov/news/news.php?feature=6362}{``Ticket to Explore JPL''}} event}{2016} \vspace{-0.1cm}
\cvitemwithcomment{}{\hspace{0.4cm}${\color{color1} \circ}\;$ Presentation to prospective students (Institute of Astronomy graduate interviews)}{2012--2014} \vspace{-0.1cm}
\cvitemwithcomment{}{\hspace{0.4cm}${\color{color1} \circ}\;$ Outreach talk at Institute of Astronomy public-observing evening: ``The Space Race''}{2012} \vspace{-0.1cm}
\cvitemwithcomment{}{\hspace{0.4cm}${\color{color1} \circ}\;$ Interactive presentation at Cambridge's Institute of Astronomy Open Day}{2011, 2013} \vspace{-0.1cm}

\subsection{Press releases}
\cvitemwithcomment{}{\hspace{0.4cm}${\color{color1} \circ}\;$ {\color{color1} \href{http://www.jpl.nasa.gov/news/news.php?feature=5505}{``\textit{Pulsar Web Could Detect Low-Frequency Gravitational Waves}''}}}{Feb 2016} \vspace{-0.1cm}
\hspace{0.71cm} First-author research (JPL press release) \\ 
\cvitemwithcomment{}{\hspace{0.4cm}${\color{color1} \circ}\;$ {\color{color1} \href{https://public.nrao.edu/news/pressreleases/2016-nanograv-sbr}{``\textit{Gravitational Wave Search Provides Insights into Galaxy Evolution and Mergers}''}}}{Apr 2016} \vspace{-0.1cm}
\hspace{0.71cm} Collaboration research  (NRAO press release) \vspace{-0.1cm}

\subsection{Media coverage}
\begin{itemize}[leftmargin=8mm]
\item Research profiled in the popular-science book {\color{color1} \href{https://books.google.com/books/about/Ripples_in_Spacetime.html?id=YicuDwAAQBAJ&hl=en}{``Ripples in Spacetime: Einstein, Gravitational Waves, and the Future of Astronomy''}} by Govert Schilling (forward by Martin Rees), Harvard University Press (2017)
\item Interviewed and quoted by \textit{Science} magazine: {\color{color1} \href{http://science.sciencemag.org/content/351/6278/1124}{``In Search of Spacetime Megawaves''}} by Daniel Clery, Science  11 Mar 2016: Vol. 351, Issue 6278, pp. 1124-1125
\item Quoted, with research featured in {\color{color1} \href{http://gizmodo.com/we-could-find-even-more-gravitational-waves-soon-with-p-1761021828}{\textit{Gizmodo}}}, {\color{color1} \href{https://www.engadget.com/2016/02/25/pulsars-gravitational-waves-black-holes/}{\textit{Engadget}}}, {\color{color1} \href{http://phys.org/news/2016-02-pulsar-web-low-frequency-gravitational.html}{\textit{Phys.org}}},  {\color{color1} \href{http://www.astronomy.com/news/2016/02/pulsar-web-could-detect-gravitational-waves}{\textit{Astronomy magazine}}}, {\color{color1} \href{http://www.universetoday.com/127562/the-future-of-gravitational-wave-astronomy-enhanced-ligo-pulsar-webs-space-interferometers-and-everything/}{\textit{Universe Today}}}
\item Collaboration research featured in {\color{color1} \href{https://www.sciencedaily.com/releases/2016/04/160405122609.htm}{\textit{Science Daily}}}, {\color{color1} \href{https://astronomynow.com/2016/04/06/gravitational-wave-search-provides-insights-into-galaxy-mergers/}{\textit{Astronomy Now} (online)}}
\end{itemize}

%\vspace{-4mm}
\section{Publications}

\cvitem{}
{
\begin{tabular}{rcl}
\textbf{Counts}: &\hspace{0.3cm} &{\textbf{24} papers published in major peer-reviewed journals}, {\textbf{1} paper in submission stage,} \\
& &{\textbf{2} other papers in conference proceedings and as arXiv documents} \\
& &{(out of which \textbf{10} are first-authored papers, and \textbf{2} papers were covered by press releases).}
\end{tabular}
}

$\bullet$ \textbf{Total number of citations:} 632 (using Google Scholar), \textbf{h-index:} 14, \textbf{i10-index:} 15

$\bullet$ \textbf{Metrics} available at Google Scholar: {\color{color1} \href{http://bit.ly/2hgVCzb}{http://bit.ly/2hgVCzb}}.

$\bullet$ \textbf{Full list of publications} listed below and at {\color{color1} \href{http://bit.ly/2fF3yGI}{http://bit.ly/2fF3yGI}}.

\subsection{Selected publications}

\begin{itemize}[leftmargin=8mm]

\item \cvitem{}{\textit{\textbf{S.~R.~Taylor}}, J.~Simon, L.~Sampson. ``\textit{Constraints On The Dynamical Environments Of Supermassive Black-hole Binaries Using Pulsar-timing Arrays}''. {\color{color1} \href{https://journals.aps.org/prl/abstract/10.1103/PhysRevLett.118.181102}{Physical Review Letters, 118, 181102 (2017)}}.}

\item \cvitem{}{\textit{\textbf{S.~R.~Taylor}}, M.~Vallisneri, J.~A.~Ellis, C.~M.~F.~Mingarelli, T.~J.~W.~Lazio, and R.~van Haasteren. ``\textit{Are We There Yet? Time to Detection of Nanohertz Gravitational Waves Based on Pulsar-timing Array Limits}''. \href{http://iopscience.iop.org/article/10.3847/2041-8205/819/1/L6/meta}{{\color{color1} Astrophys. J. Lett, 819:L6 (2016)}}.}

%\item \cvitem{}{\textit{\textbf{S.~R.~Taylor}}, C.~M.~F.~Mingarelli, J.~R.~Gair, [and 32 others]. ``\textit{Limits on Anisotropy in the Nanohertz Stochastic Gravitational Wave Background}''. \href{http://journals.aps.org/prl/abstract/10.1103/PhysRevLett.115.041101}{{\color{color1} Phys.Rev. Lett, 115(4):041101 (2015)}}.}

%\item \cvitem{}{\textit{\textbf{S.~R.~Taylor}} and J.~R.~Gair. ``\textit{Searching for anisotropic gravitational-wave backgrounds using pulsar timing arrays}''. \href{http://journals.aps.org/prd/abstract/10.1103/PhysRevD.88.084001}{{\color{color1} Phys. Rev. D, 88(8):084001 (2013)}}.}

\item \cvitem{}{\textit{\textbf{S.~R.~Taylor}}, J.~R.~Gair, and I.~Mandel. ``\textit{Cosmology using advanced gravitational-wave detectors alone}''. \href{http://journals.aps.org/prd/abstract/10.1103/PhysRevD.85.023535}{{\color{color1} Phys. Rev. D, 85(2):023535 (2012)}}.}

\end{itemize}


%$\bullet$ $5$ key publications are indicated below with preceding asterisks.
%
%\subsection{Submitted \& In Preparation} 
%\cvitem{}{\textbf{S.~R.~Taylor}, J.~Simon, L.~Sampson. ``\textit{Constraints On The Dynamical Environments Of Supermassive Black-hole Binaries Using Pulsar-timing Arrays}''. Submitted to Physical Review Letters. {\color{color1} \href{https://arxiv.org/abs/1612.02817}{arXiv:1612.02817}}.}
%\cvitem{}{\textbf{S.~R.~Taylor}, L.~Lentati, S.~Babak, P.~Brem, J.~R.~Gair, A.~Sesana, A.~Vecchio. ``\textit{All correlations must die: Assessing the significance of a stochastic gravitational-wave background in pulsar-timing arrays}''. Submitted to Physical Review D. {\color{color1} \href{https://arxiv.org/abs/1606.09180}{arXiv:1606.09180}}.}
%\cvitem{}{\textbf{S.~R.~Taylor}, R.~ van Haasteren. ``\textit{Optimized anisotropic modelling of the nanohertz gravitational-wave sky with pulsar-timing arrays}''. (In Prep.)}
%
%\subsection{Publications In Peer-reviewed International Journals} 
%\cvitem{May 2016}{G.~Desvignes, R.~N.~Caballero, L.~Lentati, [and 40 others, including \textit{\textbf{S.~R.~Taylor}}]. ``\textit{High-precision timing of 42 millisecond pulsars with the European Pulsar Timing Array}''. \href{http://mnras.oxfordjournals.org/content/458/3/3341}{{\color{color1} MNRAS, 458:3341--3380}}.}
%\cvitem{May 2016}{L.~Lentati, R.~M.~Shannon, W.~A.~Coles, [and 80 others, including \textit{\textbf{S.~R.~Taylor}}]. ``\textit{From spin noise to systematics: stochastic processes in the first International Pulsar Timing Array data release}''. \href{http://mnras.oxfordjournals.org/content/458/2/2161}{{\color{color1} MNRAS, 458:2161--2187}}.}
%\cvitem{May 2016 }{J.~P.~W.~Verbiest, L.~Lentati, G.~Hobbs, [and 89 others, including \textit{\textbf{S.~R.~Taylor}}]. ``\textit{The International Pulsar Timing Array: First data release}''. \href{http://mnras.oxfordjournals.org/content/458/2/1267}{{\color{color1} MNRAS, 458:1267--1288}}.}
%\cvitem{Apr 2016 }{Z.~Arzoumanian, A.~Brazier, S.~Burke-Spolaor, [and 48 others, including \textit{\textbf{S.~R.~Taylor}}]. ``\textit{The NANOGrav Nine-year Data Set: Limits on the Isotropic Stochastic Gravitational Wave Background}''. \href{http://iopscience.iop.org/article/10.3847/0004-637X/821/1/13/meta}{{\color{color1} Astrophys. J., 821:13}}.}
%\cvitem{Apr 2016 }{R.~N.~Caballero, K.~J.~Lee, L.~Lentati, [and 36 others, including \textit{\textbf{S.~R.~Taylor}}]. ``\textit{The noise properties of 42 millisecond pulsars from the European Pulsar Timing Array and their impact on gravitational-wave searches}''. \href{http://mnras.oxfordjournals.org/content/457/4/4421}{{\color{color1} MNRAS, 457:4421--4440}}.}
%${\color{red} \mathbf{^\Asterisk}}$\cvitem{Mar 2016 }{\textit{\textbf{S.~R.~Taylor}}, M.~Vallisneri, J.~A.~Ellis, C.~M.~F.~Mingarelli, T.~J.~W.~Lazio, and R.~van Haasteren. ``\textit{Are We There Yet? Time to Detection of Nanohertz Gravitational  Waves Based on Pulsar-timing Array Limits}''. \href{http://iopscience.iop.org/article/10.3847/2041-8205/819/1/L6/meta}{{\color{color1} Astrophys. J. Lett, 819:L6}}.}
%\cvitem{Jan 2016 }{\textit{\textbf{S.~R.~Taylor}}, E.~A.~Huerta, J.~R.~Gair, and S.~T.~McWilliams. ``\textit{Detecting Eccentric Supermassive Black Hole Binaries with Pulsar Timing Arrays: Resolvable Source Strategies}''. \href{http://iopscience.iop.org/article/10.3847/0004-637X/817/1/70/meta}{{\color{color1} Astrophys. J., 817:70}}.}
%\cvitem{Jan 2016 }{S.~Babak, A.~Petiteau, A.~Sesana, P.~Brem, P.~A.~Rosado, \textit{\textbf{S.~R.~Taylor}}, [and 30 others]. ``\textit{European Pulsar Timing Array limits on continuous gravitational waves from individual supermassive black hole binaries}''. \href{http://mnras.oxfordjournals.org/content/455/2/1665}{{\color{color1} MNRAS, 455:1665--1679}}.}
%\cvitem{Nov 2015 }{J.~R.~Gair, J.~D.~Romano, and \textit{\textbf{S.~R.~Taylor}}. ``\textit{Mapping gravitational-wave backgrounds of arbitrary polarisation using pulsar timing arrays}''. \href{http://journals.aps.org/prd/abstract/10.1103/PhysRevD.92.102003}{{\color{color1} Phys. Rev. D, 92(10):102003}}.}
%${\color{red} \mathbf{^\Asterisk}}$\cvitem{Nov 2015 }{L.~Lentati, \textit{\textbf{S.~R.~Taylor}}, C.~M.~F.~Mingarelli, [and 33 others]. ``\textit{European Pulsar Timing Array limits on an isotropic stochastic gravitational-wave background}''. \href{http://mnras.oxfordjournals.org/content/453/3/2576}{{\color{color1} MNRAS, 453:2576--2598}}.}
%\cvitem{Sep 2015 }{E.~A.~Huerta, S.~T.~McWilliams, J.~R.~Gair, and \textit{\textbf{S.~R.~Taylor}}. ``\textit{Detection of eccentric supermassive black hole binaries with pulsar timing arrays: Signal-to-noise ratio calculations}''. \href{http://journals.aps.org/prd/abstract/10.1103/PhysRevD.92.063010}{{\color{color1} Phys. Rev. D, 92(6):063010}}.}
%\cvitem{Aug 2015 }{J.~D.~Romano, \textit{\textbf{S.~R.~Taylor}}, N.~J.~Cornish, J.~Gair, C.~M.~F.~Mingarelli, and R.~van Haasteren. ``\textit{Phase-coherent mapping of gravitational-wave backgrounds using ground-based laser interferometers}'', \href{http://journals.aps.org/prd/abstract/10.1103/PhysRevD.92.042003}{{\color{color1} Phys. Rev. D, 92(4):042003}}.}
%${\color{red} \mathbf{^\Asterisk}}$\cvitem{Jul 2015}{\textit{\textbf{S.~R.~Taylor}}, C.~M.~F.~Mingarelli, J.~R.~Gair, [and 32 others]. ``\textit{Limits on Anisotropy in the Nanohertz Stochastic Gravitational Wave Background}''. \href{http://journals.aps.org/prl/abstract/10.1103/PhysRevLett.115.041101}{{\color{color1} Phys.Rev. Lett, 115(4):041101}}.}
%\cvitem{Mar 2015}{C.~J.~Moore, \textit{\textbf{S.~R.~Taylor}}, and J.~R.~Gair. ``\textit{Estimating the sensitivity of pulsar timing arrays}'', \href{http://iopscience.iop.org/article/10.1088/0264-9381/32/5/055004/meta}{{\color{color1} Classical and Quantum Gravity, 32(5):055004}}.}
%\cvitem{Nov 2014}{\textit{\textbf{S.~R.~Taylor}}, J.~Ellis, and J.~Gair. ``\textit{Accelerated Bayesian model-selection and parameter-estimation in continuous gravitational-wave searches with pulsar-timing arrays}''. \href{http://journals.aps.org/prd/abstract/10.1103/PhysRevD.90.104028}{{\color{color1} Phys. Rev. D, 90(10):104028}}.}
%\cvitem{Oct 2014}{J.~Gair, J.~D.~Romano, \textit{\textbf{S.~R.~Taylor}}, and C.~M.~F.~Mingarelli. ``\textit{Mapping gravitational-wave backgrounds using methods from CMB analysis: Application to pulsar timing arrays}''. \href{http://journals.aps.org/prd/abstract/10.1103/PhysRevD.90.082001}{{\color{color1} Phys. Rev. D, 90(8):082001}}.}
%\cvitem{Aug 2014}{R.~M.~Shannon, S.~Chamberlin, N.~J.~Cornish, J.~A.~Ellis, C.~M.~F.~Mingarelli, D.~Perrodin, P.~Rosado, A.~Sesana, \textit{\textbf{S.~R.~Taylor}}, [and 14 others]. ``\textit{Summary of Session C1: pulsar timing arrays}''. \href{http://link.springer.com/article/10.1007\%2Fs10714-014-1765-4}{{\color{color1} General Relativity and Gravitation, 46:1765}}.}
%${\color{red} \mathbf{^\Asterisk}}$\cvitem{Oct 2013}{\textit{\textbf{S.~R.~Taylor}} and J.~R.~Gair. ``\textit{Searching for anisotropic gravitational-wave backgrounds using pulsar timing arrays}''. \href{http://journals.aps.org/prd/abstract/10.1103/PhysRevD.88.084001}{{\color{color1} Phys. Rev. D, 88(8):084001}}.}
%\cvitem{May 2013}{L.~Lentati, P.~Alexander, M.~P.~Hobson, \textit{\textbf{S.~R.~Taylor}}, J.~Gair, S.~T.~Balan, and R.~van Haasteren. ``\textit{Hyper-efficient model-independent Bayesian method for the analysis  of  pulsar timing data}''. \href{http://journals.aps.org/prd/abstract/10.1103/PhysRevD.87.104021}{{\color{color1} Phys. Rev. D, 87(10):104021}}.}
%\cvitem{Feb 2013 }{\textit{\textbf{S.~R.~Taylor}}, J.~R.~Gair, and L.~Lentati. ``\textit{Weighing the evidence for a gravitational-wave background in the first International Pulsar Timing Array data challenge}''. \href{http://journals.aps.org/prd/abstract/10.1103/PhysRevD.87.044035}{{\color{color1} Phys. Rev. D, 87(4):044035}}.}
%\cvitem{Jul 2012}{\textit{\textbf{S.~R.~Taylor}} and J.~R.~Gair. ``\textit{Cosmology with the lights off: Standard sirens in the Einstein Telescope era}''. \href{http://journals.aps.org/prd/abstract/10.1103/PhysRevD.86.023502}{{\color{color1} Phys. Rev. D, 86(2):023502}}.}
%${\color{red} \mathbf{^\Asterisk}}$\cvitem{Jan 2012}{\textit{\textbf{S.~R.~Taylor}}, J.~R.~Gair, and I.~Mandel. ``\textit{Cosmology using advanced gravitational-wave detectors alone}''. \href{http://journals.aps.org/prd/abstract/10.1103/PhysRevD.85.023535}{{\color{color1} Phys. Rev. D, 85(2):023535}}.}
%
%\pagebreak

%\vspace{-4mm}
\section{Presentations}

$\bullet$ \textbf{$43$ oral presentations (of which $18$ were invited), with $8$ conference leadership roles.}

$\bullet$ Recent presentations are available to view at {\color{color1} \href{https://speakerdeck.com/stevertaylor}{https://speakerdeck.com/stevertaylor}}.

$\bullet$ \textbf{Full list of presentations} listed below and at ...

\subsection{Selected presentations}

\begin{itemize}[leftmargin=8mm]

\item (Invited): \textit{Constraining the physics of the final parsec of supermassive black-hole binary evolution} \\ Astronomy Colloquium, Swinburne University of Technology, Melbourne, Australia, Feb 2017

\item (Invited): \textit{Probing the nanohertz GW landscape with PTAs: a status report} \\ Gravitational Wave Physics \& Astronomy Workshop (GWPAW), Annecy, France, May 2017.

\item (Invited): \textit{Gravitational-wave data-analysis techniques for pulsar-timing arrays} \\ IPTA conference, Stellenbosch, South Africa, Jun 2016.

\end{itemize}

%\subsection{Invited Talks} 
%\cvitem{Oct 2016}{\textit{Astrophysical inference of supermassive black-hole binaries with pulsar-timing arrays}, Leonard E. Parker Center seminar, University of Wisconsin--Milwaukee, Milwaukee WI, USA}
%\cvitem{Oct 2016}{\textit{Astrophysical inference of supermassive black-hole binaries with pulsar-timing arrays}, CIERA seminar, Northwestern University, Evanston IL, USA}
%\cvitem{Oct 2016}{\textit{New data-analysis approaches for gravitational-wave searches with pulsar-timing arrays}, Montana State University seminar, Bozeman MT, USA}
%\cvitem{Jul 2016}{\textit{New horizons in gravitational-wave astronomy with pulsar-timing arrays}, Armagh Observatory seminar, Armagh, UK}
%\cvitem{Jul 2016}{\textit{Probing the final-parsec problem with pulsar-timing arrays}, Anton Pannekoek Institutt seminar, University of Amsterdam, Amsterdam, Netherlands}
%\cvitem{Jul 2016}{\textit{Probing the final-parsec problem with pulsar-timing arrays}, Radboud University astrophysics seminar, Radboud, Netherlands}
%\cvitem{Jun 2016}{\textit{Gravitational-wave data-analysis techniques for pulsar-timing arrays}, IPTA conference, Stellenbosch, South Africa}
%\cvitem{Mar 2016}{\textit{Sources of nanohertz gravitational-waves for pulsar-timing array searches}, NANOGrav student workshop, Caltech, Pasadena CA, USA}
%\cvitem{Dec 2015}{\textit{Prospects for near future detection and astrophysical inference with PTAs}, Gravitational-wave group seminar, University of Birmingham, UK}
%\cvitem{Dec 2015}{\textit{Prospects for near future detection and astrophysical inference with PTAs}, Statistics group seminar (School of Mathematics), University of Edinburgh, UK}
%\cvitem{Dec 2015}{\textit{Prospects for near future detection and astrophysical inference with PTAs}, CaJAGWR seminar, California Institute of Technology}
%\cvitem{May 2013}{\textit{Searching For Anisotropic Gravitational-wave Backgrounds Using Pulsar Timing Arrays}, Albert Einstein Institute (AEI) GW seminar, Hanover}
%\cvitem{Dec 2012}{\textit{Weighing the evidence for a gravitational-wave background}, Gravitational-wave group seminar, University of Birmingham, UK}
%
%\subsection{Contributed Presentations} 
%\cvitem{Oct 2016}{\textit{Optimized gravitational-wave sky mapping with pulsar-timing arrays}, NANOGrav Fall Meeting 2016, NCSA, Urbana-Champaign IL, USA}
%\cvitem{May 2016}{\textit{Carrying the physics of supermassive black-hole binary evolution into pulsar-timing array searches}, EPTA meeting, Bielefeld, Germany}
%\cvitem{Apr 2016}{\textit{Are we there yet? Time to detection of nanohertz gravitational waves}, American Physical Society meeting, Salt Lake City UT, USA}
%\cvitem{Mar 2016}{\textit{Carrying the physics of supermassive black-hole binary evolution into pulsar-timing array searches}, NANOGrav meeting, Caltech, Pasadena CA, USA}
%\cvitem{Oct 2015}{\textit{Are we there yet? Time to detection of nanohertz gravitational waves}, NANOGrav meeting, McGill University, Montreal, Canada}
%\cvitem{Jun 2015}{\textit{Eccentric supermassive black-hole binary signals in pulsar-timing data}, European Pulsar Timing Array meeting, Bonn, Germany}
%\cvitem{Apr 2015}{\textit{Eccentric supermassive black-hole binary signals in pulsar-timing data}, American Physical Society meeting, Baltimore MD, USA}
%\cvitem{Feb 2015}{\textit{Eccentric supermassive black-hole binary signals in pulsar-timing data}, NANOGrav meeting, Arecibo, Puerto Rico}
%\cvitem{Jan 2015}{\textit{Exploring the cosmos with gravitational waves}, American Astronomical Society meeting, Seattle WA, USA}
%\cvitem{Nov 2014}{\textit{EPTA constraints on gravitational-wave anisotropy}, European Pulsar Timing Array meeting, Cambridge, UK}
%\cvitem{Jun 2014}{\textit{EPTA and IPTA searches for gravitational-wave background anisotropy}, International Pulsar Timing Array meeting, Banff, Canada}
%\cvitem{May 2014}{\textit{EPTA limits on gravitational-wave anisotropy}, European Pulsar Timing Array meeting, Astron, Netherlands}
%\cvitem{Oct 2013}{\textit{The pulsar-term in PTA continuous-wave searches: a blessing and a curse}, European Pulsar Timing Array meeting, Pula, Sardinia}
%\cvitem{Jul 2013}{\textit{Probing anisotropy of the GW background with pulsar timing arrays}, 20th International Conference on General Relativity and Gravitation and 10th Amaldi Conference on Gravitational Waves, Warsaw}
%\cvitem{Jun 2013}{\textit{The first PTA search pipeline for anisotropy in the GW background}, International Pulsar Timing Array meeting, Krabi, Thailand}
%\cvitem{Apr 2013}{\textit{Searching For Anisotropic Gravitational-wave Backgrounds Using Pulsar Timing Arrays}, European Pulsar Timing Array meeting, l'Observatoire de Paris, Paris}
%\cvitem{Feb 2013}{\textit{Weighing the evidence for a gravitational-wave background}, Institute of Astronomy seminar, University of Cambridge}
%\cvitem{Nov 2012}{\textit{Weighing the evidence for a gravitational-wave background}, European Pulsar Timing Array meeting, Albert Einstein Institute (AEI), Potsdam, Germany}
%\cvitem{Jun 2012}{\textit{Milestones in Spacetime: Double Neutron-Star Binaries as Gravitational-Wave Standard Sirens}, Institute of Astronomy seminar, University of Cambridge, UK}
%%\cvitem{Feb 2012}{\textit{Hubble without the Hubble: Cosmology using advanced gravitational-wave detectors alone}, Gravitational-Wave Meeting, Institut de Ci�ncies de l'Espai, Barcelona, Spain}
%
%\subsection{Posters} 
%\cvitem{Aug 2015}{\textit{Galactic environment effects on gravitational wave signals in pulsar timing arrays}, Postdoc Research Day, NASA Jet Propulsion Laboratory}
%\cvitem{Aug 2012}{\textit{Cosmology without EM counterparts: Standard sirens in the advanced era and beyond}, Rattle and Shine, KITP Santa Barbara}
%\cvitem{Dec 2011}{\textit{Cosmology using advanced gravitational-wave detectors alone}, Graduate Student Conference 2011, Cavendish Laboratory, University of Cambridge}

%----------------------------------------------------------------------------------------
%   COMPUTER SKILLS SECTION
%----------------------------------------------------------------------------------------

\section{Skills}
\begin{itemize}[leftmargin=8mm]
\item \textbf{OS:} Linux/UNIX, Windows
\item \textbf{Programming:} \textsc{Python} (advanced), C/C++, UNIX shell scripting, HTML, High-performance cluster computing, GPU programming (CUDA C, PyCUDA)
\item \textbf{Typography:} \LaTeX, Bibtex, Microsoft Office, Pages, OpenOffice 
\item \textbf{Scientific:} Mathematica, Matlab, \textsc{Python}
\end{itemize}

%----------------------------------------------------------------------------------------
%  REFERENCES
%----------------------------------------------------------------------------------------

\section{References (\textit{more available upon request})} 
\vspace{10pt}
\begin{cvcolumns}
\cvcolumn{}{\cvreference{Dr~Jonathan~R.~Gair [\textit{PhD advisor}]}
    {Reader, School of Mathematics}
    {James Clerk Maxwell Building}
    {Peter Guthrie Tait Road}
    {University of Edinburgh}
    {Edinburgh UK, EH9 3FD}
    {}
    {J.Gair@ed.ac.uk}
    {+44 (0) 131 650 4899}
    }
\cvcolumn{}{\cvreference{Prof. Xavier Siemens}
    {}%Chair, NANOGrav}
    {Director, NANOGrav Physics Frontier Center}
    {Associate Professor}
    {Department of Physics}
    {University of Wisconsin--Milwaukee}
    {Milwaukee,  WI 53201}
    {siemens@gravity.phys.uwm.edu}
    {+1 (414) 229 6439}
    }
   \end{cvcolumns}
 
\vspace{5pt}
\begin{cvcolumns}
\cvcolumn{}{\cvreference{Prof.~Neil~J.~Cornish}
    {Professor of Physics}
    {eXtreme Gravity Institute}
    {Department of Physics}
    {Montana State University}
    {Bozeman, MT 59717}
    {}
    {ncornish@montana.edu}
    {+1 (406) 994 7986}
    }
%    \cvcolumn{}{\cvreference{Prof. Alberto Vecchio}
%    {Head of Group}
%    {School of Physics \& Astronomy}
%    {University of Birmingham}
%    {Edgbaston}
%    {Birmingham UK, B15 2TT}
%    {}
%    {av@star.sr.bham.ac.uk}
%    {+44 (0) 1214 146447}
%    }
    \end{cvcolumns}
    
%\vspace{20pt}
%\begin{cvcolumns}
%\cvcolumn{}{\cvreference{Dr~T.~Joseph~W.~Lazio}
%    {Chief Scientist}
%    {Interplanetary Network Directorate}
%    {Jet Propulsion Laboratory}
%    {California Institute of Technology}
%    {4800 Oak Grove Drive}
%    {Pasadena, CA 91109}
%    {Joseph.Lazio@jpl.nasa.gov}
%    {+1 (818) 354 4198}
%    }
%    \cvcolumn{}{\cvreference{Dr~Michele~Vallisneri}
%    {Director's Fellow}
%    {Jet Propulsion Laboratory}
%    {California Institute of Technology}
%    {4800 Oak Grove Drive}
%    {Pasadena, CA 91109}
%    {}
%    {michele.vallisneri@jpl.nasa.gov}
%    {+1 (818) 395 9073}
%    }
%    \end{cvcolumns}

\end{document}